\appendix
\section{Rectangle Size Derivation} \label{appendix:nyquist}

As mentioned in Section~\ref{section:illumination} of the main paper, a ray that reaches the environment map is assigned a color taken as the average color over an axis-aligned rectangle in spherical coordinates, where the shape of the rectangle depends on the ray's direction and the material's roughness at the ray's origin. %The area and aspect ratio of the sample rectangle are adapted to each ray. 
We modify the derivation of the area of the rectangle from GPU Gems~\cite{colbert2007gpu}. Let $N$ be the number of samples, $p(\omegai)$ be the probability density function of a given sample direction $\omegai$ for viewing direction $\omegao$, and let $H$ and $W$ be the height and width of the environment map (\ie its polar and azimuthal resolutions). The density $d(\omegai)$ of environment map pixels at a given direction must be inversely proportional to the Jacobian's determinant, $\sin\theta_i$, and it must also satisfy: 
\begin{equation}
    HW = \int_0^{2\pi}\int_0^\pi d(\omegai) \sin\theta_i d\theta_i d\phi_i,
\end{equation}
and therefore:
\begin{equation}
    d(\omegai) = \frac{HW}{2\pi^2\sin\theta_i}.
\end{equation}

The number of pixels per sample, which is the area of the rectangle, is then the total solid angle per sample, $Np(\omegai)$ multiplied by the number of pixels per solid angle:
\begin{equation}
    \Delta\theta\cdot \Delta\phi = \frac{Np(\omegai)}{d(\omegai)},
\end{equation}
where $\Delta\theta$ is the polar size of the rectangle, and $\Delta\phi$ is its azimuthal size, \ie the rectangle is $\Delta\theta\times\Delta\phi$, in equirectangular coordinates.
%We multiply the pixels per a solid angle, $1/d(\omegai)$, by the solid angle per sample, $N p(\omegai, \omegao)$, to get the pixels per a sample, which is our desired rectangle size.

As mentioned in Section~\ref{section:illumination} of the main paper, the aspect ratio of the rectangle is set to:
\begin{equation}
    \frac{\Delta\theta}{\Delta\phi} = \sin\theta_i,
\end{equation}
which yields:
\begin{align}
    \Delta\theta &= \sqrt{2\pi^2 \frac{N}{HW} p(\omegai)}\cdot \sin\theta_i ,\\
    \Delta\phi &= \sqrt{2\pi^2 \frac{N}{HW} p(\omegai)}.
\end{align}

%The ratio of width to height based on the distortion of the projection can then be used to calculate the width and height of the rectangle corresponding to each sample.

\section{BSDF Neural Network Parameterization} \label{appendix:parameterization}

Once we have sampled the incoming light directions $\omegai$ and their respective values $L(\point, \omegai)$, we transform them into the local shading frame to calculate the value of the neural shading network $h$. We parameterize the neural network with 2 hidden layers of width 64 as $h(\point, \omegao, \omegai, \uvn)$, where $\point$ is the position, $\omegao, \omegai$ are the outgoing and incoming light directions, respectively, and $\uvn$ is the normal. However, rather than feeding $\omegao$ and $\omegai$ to the network directly, we follow the schema laid out by Rusinkiewicz~\cite{rusinkiewicz1998new} and parameterize the input using the halfway vector $\boldsymbol{\hat{h}}$ and difference vector $\boldsymbol{\hat{d}}$ within the local shading frame $F(\uvn)$, which takes the world space to a frame of reference in which the normal vector points upwards:
\begin{align}
    T&={[0, 0, 1]}^\top \times \uvn \\
    F(\uvn) &= \begin{bmatrix}
    T, & \uvn\times T, & \uvn
    \end{bmatrix}^\top \\
    \boldsymbol{\hat{h}} &= F(\uvn) \frac{\omegai+\omegao}{\|\omegai+\omegao\|_2} \\
    \boldsymbol{\hat{d}} &= F(\boldsymbol{\hat{h}}) \omegai
\end{align}
where $\times$ is the cross product. Finally, we encode these two directions using spherical harmonics up to degree 4 (as done in Ref-NeRF~\cite{verbin2021ref} for encoding view directions), concatenate the feature vector $\feat$ from the field at point $\point$, and pass this as input to the network $h$. 





\section{Optimization and Architecture}\label{appendix:optimization}

To calculate the normal vectors of the density field, we apply a finite difference kernel, convolved with a $3\times 3$ Gaussian smoothing kernel with $\sigma=1$, then linearly interpolate between samples to get the resulting gradient in the 3D volume.
We supervise our method using photometric loss, along with the orientation loss of Equation~\ref{eqn:normal_penalty}. Like TensoRF, we use a learning rate of $0.02$ for the rank $1$ and $2$ tensor components, and a learning rate of $10^{-3}$ for everything else. We use Adam~\cite{kingma2014adam} with $\beta_1=0.9,\beta_2=0.99, \varepsilon=10^{-15}$. %We decay the learning rate log-linearly in the same way as Ref-NeRF~\cite{verbin2021ref}.
Similar to Ref-NeRF~\cite{verbin2021ref}, we use log-linear learning rate decay with a total decay of $d_w = 10^{-3}$ and a warmup of $N_w = 100$ steps and a decay multiplier of $m_w = 0.1$ over $N_T=3\cdot 10^4$ total iterations. This gives us the following formula for the learning rate multiplier for some iteration $i$:
\begin{equation}
    \left[m_w + (1-m_w) \sin\frac\pi2 \text{clip}\left(\frac{i}{N_w}, 0, 1\right)\right] e^{i/N_T}\log(d_w)
\end{equation}

We initialize the environment map to a constant value of $0.5$. 
Finally, we upsample the resolution of TensoRF from $32^3$ up to $300^3$ cube-root-linearly at steps $500, 1000, 2000, 3000, 4000, 5500, 7000$, and don't shrink the volume to fit the model.

To further reduce the variance of the estimated value of the rendering equation (see Equation~\ref{eqn:montecarlo}), we use quasi-random sampling sequences. Specifically, we use a Sobol sequence~\cite{sable1967} with Owens scrambling~\cite{owen1995randomly}, which gives the procedural sequence necessary for assigning an arbitrary number of secondary ray samples to each primary ray sample. We then apply Cranley-Patterson rotation~\cite{cranley1976randomization} to avoid needing to redraw samples.


\section{Additional Results}
\label{appendix:results}
Tables 2-5 contain full per-scene metrics for our method as well as ablations and baselines. Visual comparisons are also provided in Figures 7-17.

\begin{table*}[]
\resizebox{\linewidth}{!}{
\begin{tabular}{l|rrrrrr|rrrrrrrr}
\rowcolor[HTML]{FFFFFF} 
PSNR $\uparrow$                             & \multicolumn{1}{l}{\cellcolor[HTML]{FFFFFF}teapot} & \multicolumn{1}{l}{\cellcolor[HTML]{FFFFFF}toaster} & \multicolumn{1}{l}{\cellcolor[HTML]{FFFFFF}car} & \multicolumn{1}{l}{\cellcolor[HTML]{FFFFFF}ball} & \multicolumn{1}{l}{\cellcolor[HTML]{FFFFFF}coffee} & \multicolumn{1}{l|}{\cellcolor[HTML]{FFFFFF}helmet} & \multicolumn{1}{l}{\cellcolor[HTML]{FFFFFF}chair} & \multicolumn{1}{l}{\cellcolor[HTML]{FFFFFF}lego} & \multicolumn{1}{l}{\cellcolor[HTML]{FFFFFF}materials} & \multicolumn{1}{l}{\cellcolor[HTML]{FFFFFF}mic} & \multicolumn{1}{l}{\cellcolor[HTML]{FFFFFF}hotdog} & \multicolumn{1}{l}{\cellcolor[HTML]{FFFFFF}ficus} & \multicolumn{1}{l}{\cellcolor[HTML]{FFFFFF}drums} & \multicolumn{1}{l}{\cellcolor[HTML]{FFFFFF}ship} \\ \hline
\rowcolor[HTML]{FFFFFF} 
PhySG$^1$                                   & 35.83                                              & 18.59                                               & 24.40                                           & 27.24                                            & 23.71                                              & 27.51                                               & 21.87                                             & 17.10                                            & 18.02                                                 & 19.16                                           & 24.49                                              & 15.25                                             & 14.35                                             & 18.06                                            \\
\rowcolor[HTML]{FFFFFF} 
NVDiffRec$^1$                               & 40.13                                              & 24.10                                               & 27.13                                           & 30.77                                            & 30.58                                              & 26.66                                               & 32.03                                             & 29.07                                            & 25.03                                                 & 30.72                                           & 33.05                                              & \cellcolor[HTML]{FFD9B3}31.18                     & 24.53                                             & 24.68                                            \\
\rowcolor[HTML]{FFFFFF} 
NVDiffRecMC$^1$                             & 37.91                                              & 21.93                                               & 25.84                                           & 28.89                                            & 29.06                                              & 25.57                                               & 28.13                                             & 26.46                                            & 25.64                                                 & 29.03                                           & 30.56                                              & 25.32                                             & 22.78                                             & 18.59                                            \\
\rowcolor[HTML]{FFB3B3} 
\cellcolor[HTML]{FFFFFF}Ref-NeRF$^2$        & 47.90                                              & \cellcolor[HTML]{FFFFFF}25.70                       & 30.82                                           & 47.46                                            & 34.21                                              & \cellcolor[HTML]{FFFFB4}29.68                       & 35.83                                             & 36.25                                            & 35.41                                                 & 36.76                                           & 37.72                                              & 33.91                                             & 25.79                                             & 30.28                                            \\
\rowcolor[HTML]{FFFFFF} 
Ours, no integral image                     & 42.61                                              & 18.36                                               & 25.32                                           & 21.70                                            & 31.15                                              & 24.82                                               & 30.35                                             & 30.16                                            & 25.62                                                 & 30.03                                           & 33.34                                              & 28.44                                             & 24.04                                             & 25.78                                            \\
\rowcolor[HTML]{FFFFFF} 
Ours, analytical derivative                 & 43.57                                              & 21.57                                               & 27.72                                           & 22.75                                            & 31.08                                              & 28.61                                               & 30.49                                             & 30.23                                            & 28.70                                                 & 31.19                                           & 33.55                                              & 27.83                                             & 24.15                                             & 25.40                                            \\
\rowcolor[HTML]{FFFFB4} 
\cellcolor[HTML]{FFFFFF}Ours, single bounce & 45.23                                              & \cellcolor[HTML]{FFD9B3}26.91                       & 30.13                                           & 38.38                                            & 31.39                                              & \cellcolor[HTML]{FFD9B3}34.32                       & \cellcolor[HTML]{FFD9B3}32.57                     & 32.83                                            & 30.92                                                 & \cellcolor[HTML]{FFD9B3}32.49                   & 35.07                                              & 29.24                                             & \cellcolor[HTML]{FFD9B3}24.99                     & 27.32                                            \\
\rowcolor[HTML]{FFFFFF} 
Ours, no neural                             & 45.21                                              & \cellcolor[HTML]{FFFFB4}25.73                       & 29.03                                           & 37.41                                            & 30.99                                              & 29.63                                               & 30.62                                             & 31.00                                            & 29.37                                                 & 31.29                                           & 33.88                                              & 28.10                                             & 24.52                                             & 26.44                                            \\
\rowcolor[HTML]{FFD9B3} 
\cellcolor[HTML]{FFFFFF}Ours                & 45.29                                              & \cellcolor[HTML]{FFB3B3}27.52                       & 30.28                                           & 38.41                                            & 31.47                                              & \cellcolor[HTML]{FFB3B3}34.38                       & \cellcolor[HTML]{FFFFB4}32.27                     & 32.98                                            & 31.19                                                 & \cellcolor[HTML]{FFFFB4}32.41                   & 35.23                                              & \cellcolor[HTML]{FFFFFF}29.24                     & \cellcolor[HTML]{FFFFB4}24.96                     & 27.37                                           
\end{tabular}
}
\small{$^1$ requires object masks during training. ~~ $^2$ view synthesis method, not inverse rendering. ~~ Red is best, followed by orange, then yellow. }
\caption{\textbf{PSNR Results on the \emph{Shiny Blender} dataset from Ref-NeRF \cite{verbin2021ref} and \emph{Blender} dataset from NeRF \cite{mildenhall2021nerf}.}}
\label{tab:psnrs}
\end{table*}

\begin{table*}[]
\resizebox{\linewidth}{!}{
\begin{tabular}{l|rrrrrr|rrrrrrrr}
\rowcolor[HTML]{FFFFFF} 
SSIM $\uparrow$                             & \multicolumn{1}{l}{\cellcolor[HTML]{FFFFFF}teapot} & \multicolumn{1}{l}{\cellcolor[HTML]{FFFFFF}toaster} & \multicolumn{1}{l}{\cellcolor[HTML]{FFFFFF}car} & \multicolumn{1}{l}{\cellcolor[HTML]{FFFFFF}ball} & \multicolumn{1}{l}{\cellcolor[HTML]{FFFFFF}coffee} & \multicolumn{1}{l|}{\cellcolor[HTML]{FFFFFF}helmet} & \multicolumn{1}{l}{\cellcolor[HTML]{FFFFFF}chair} & \multicolumn{1}{l}{\cellcolor[HTML]{FFFFFF}lego} & \multicolumn{1}{l}{\cellcolor[HTML]{FFFFFF}materials} & \multicolumn{1}{l}{\cellcolor[HTML]{FFFFFF}mic} & \multicolumn{1}{l}{\cellcolor[HTML]{FFFFFF}hotdog} & \multicolumn{1}{l}{\cellcolor[HTML]{FFFFFF}ficus} & \multicolumn{1}{l}{\cellcolor[HTML]{FFFFFF}drums} & \multicolumn{1}{l}{\cellcolor[HTML]{FFFFFF}ship} \\ \hline
\rowcolor[HTML]{FFFFFF} 
PhySG$^1$                                   & .990                                               & .805                                                & .910                                            & .947                                             & .922                                               & .953                                                & .890                                              & .812                                             & .837                                                  & .904                                            & .894                                               & .861                                              & .823                                              & .756                                             \\
\rowcolor[HTML]{FFFFFF} 
NVDiffRec$^1$                               & .993                                               & .898                                                & .938                                            & .949                                             & .959                                               & .931                                                & \cellcolor[HTML]{FFD9B3}.969                      & .949                                             & .923                                                  & \cellcolor[HTML]{FFFFB4}.977                    & \cellcolor[HTML]{FFD9B3}.973                       & \cellcolor[HTML]{FFD9B3}.970                      & .916                                              & \cellcolor[HTML]{FFFFB4}.833                     \\
\rowcolor[HTML]{FFFFFF} 
NVDiffRecMC$^1$                             & .990                                               & .842                                                & .913                                            & .849                                             & .942                                               & .877                                                & .932                                              & .909                                             & .911                                                  & .961                                            & .945                                               & .937                                              & .906                                              & .732                                             \\
\rowcolor[HTML]{FFB3B3} 
\cellcolor[HTML]{FFFFFF}Ref-NeRF$^2$        & .998                                               & .922                                                & .955                                            & .995                                             & .974                                               & \cellcolor[HTML]{FFFFB4}.958                        & .984                                              & .981                                             & .983                                                  & .992                                            & .984                                               & .983                                              & .937                                              & .880                                             \\
\rowcolor[HTML]{FFFFFF} 
Ours, no integral image                     & .994                                               & .734                                                & .895                                            & .753                                             & .959                                               & .880                                                & .946                                              & .946                                             & .896                                                  & .962                                            & .954                                               & .953                                              & .905                                              & .794                                             \\
\rowcolor[HTML]{FFFFFF} 
Ours, analytical derivative                 & \cellcolor[HTML]{FFFFB4}.995                       & .798                                                & .925                                            & .790                                             & .959                                               & .930                                                & .948                                              & .943                                             & .936                                                  & .972                                            & .958                                               & .950                                              & .910                                              & .787                                             \\
\cellcolor[HTML]{FFFFFF}Ours, single bounce & \cellcolor[HTML]{FFD9B3}.996                       & \cellcolor[HTML]{FFFFB4}.909                        & \cellcolor[HTML]{FFD9B3}.951                    & \cellcolor[HTML]{FFD9B3}.983                     & \cellcolor[HTML]{FFD9B3}.962                       & \cellcolor[HTML]{FFB3B3}.971                        & \cellcolor[HTML]{FFFFB4}.964                      & \cellcolor[HTML]{FFD9B3}.966                     & \cellcolor[HTML]{FFFFB4}.957                          & \cellcolor[HTML]{FFD9B3}.978                    & \cellcolor[HTML]{FFFFB4}.969                       & \cellcolor[HTML]{FFFFB4}.959                      & \cellcolor[HTML]{FFD9B3}.922                      & \cellcolor[HTML]{FFD9B3}.835                     \\
\rowcolor[HTML]{FFFFFF} 
Ours, no neural                             & \cellcolor[HTML]{FFD9B3}.996                       & .903                                                & \cellcolor[HTML]{FFFFB4}.945                                            & \cellcolor[HTML]{FFFFB4}.980                                             & .959                                               & .947                                                & .949                                              & .952                                             & .945                                                  & .972                                            & .960                                               & .954                                              & .916                                              & .816                                             \\
\cellcolor[HTML]{FFFFFF}Ours                & \cellcolor[HTML]{FFD9B3}.996                       & \cellcolor[HTML]{FFD9B3}.917                        & \cellcolor[HTML]{FFD9B3}.951                    & \cellcolor[HTML]{FFD9B3}.983                     & \cellcolor[HTML]{FFFFB4}.960                       & \cellcolor[HTML]{FFD9B3}.969                        & \cellcolor[HTML]{FFFFFF}.956                      & \cellcolor[HTML]{FFFFB4}.963                     & \cellcolor[HTML]{FFD9B3}.959                          & \cellcolor[HTML]{FFFFFF}.977                    & \cellcolor[HTML]{FFFFFF}.964                       & \cellcolor[HTML]{FFFFFF}.952                      & \cellcolor[HTML]{FFFFB4}.917                      & \cellcolor[HTML]{FFFFFF}.828                    
\end{tabular}
}
\small{$^1$ requires object masks during training. ~~ $^2$ view synthesis method, not inverse rendering.~~ Red is best, followed by orange, then yellow. }
\caption{\textbf{SSIM Results on the \emph{Shiny Blender} dataset from Ref-NeRF \cite{verbin2021ref} and \emph{Blender} dataset from NeRF \cite{mildenhall2021nerf}.}}
\label{tab:ssims}
\end{table*}

\begin{table*}[]
\resizebox{\linewidth}{!}{
\begin{tabular}{l|rrrrrr|rrrrrrrr}
LPIPS $\downarrow$                          & \multicolumn{1}{l}{teapot}   & \multicolumn{1}{l}{toaster}  & \multicolumn{1}{l}{car}      & \multicolumn{1}{l}{ball}     & \multicolumn{1}{l}{coffee}   & \multicolumn{1}{l|}{helmet}  & \multicolumn{1}{l}{chair}    & \multicolumn{1}{l}{lego}     & \multicolumn{1}{l}{materials} & \multicolumn{1}{l}{mic}      & \multicolumn{1}{l}{hotdog}   & \multicolumn{1}{l}{ficus}    & \multicolumn{1}{l}{drums}    & \multicolumn{1}{l}{ship}     \\ \hline
\rowcolor[HTML]{FFFFFF} 
PhySG$^1$                                   & .022                         & .194                         & .091                         & .179                         & .150                         & .089                         & .122                         & .208                         & .182                          & .108                         & .163                         & .144                         & .188                         & .343                         \\
\rowcolor[HTML]{FFFFFF} 
NVDiffRec$^1$                               & .022                         & .180                         & .057                         & .194                         & .097                         & .134                         & \cellcolor[HTML]{FFD9B3}.027 & .037                         & .104                          & .033                         & \cellcolor[HTML]{FFFFB4}.038 & \cellcolor[HTML]{FFD9B3}.030 & .070                         & .208                         \\
\rowcolor[HTML]{FFFFFF} 
NVDiffRecMC$^1$                             & .029                         & .243                         & .086                         & .346                         & .131                         & .215                         & .080                         & .075                         & .096                          & .057                         & .089                         & .076                         & .096                         & .319                         \\
\rowcolor[HTML]{FFB3B3} 
\cellcolor[HTML]{FFFFFF}Ref-NeRF$^2$        & .004                         & .095                         & \cellcolor[HTML]{FFFFFF}.041 & \cellcolor[HTML]{FFFFFF}.059 & \cellcolor[HTML]{FFFFFF}.078 & \cellcolor[HTML]{FFFFB4}.075 & .017                         & \cellcolor[HTML]{FFB3B3}.018 & .022                          & .007                         & .022                         & .019                         & .059                         & \cellcolor[HTML]{FFD9B3}.139 \\
\rowcolor[HTML]{FFFFFF} 
Ours, no integral image                     & .013                         & .285                         & .077                         & .399                         & \cellcolor[HTML]{FFD9B3}.065 & .180                         & .055                         & .031                         & .074                          & .042                         & .051                         & .039                         & .077                         & .180                         \\
\rowcolor[HTML]{FFFFFF} 
Ours, analytical derivative                 & .011                         & .235                         & .053                         & .353                         & .071                         & .118                         & .052                         & .031                         & .048                          & .028                         & .047                         & .043                         & .075                         & .190                         \\
\cellcolor[HTML]{FFFFFF}Ours, single bounce & \cellcolor[HTML]{FFD9B3}.008 & \cellcolor[HTML]{FFFFB4}.114 & \cellcolor[HTML]{FFB3B3}.033 & \cellcolor[HTML]{FFD9B3}.047 & \cellcolor[HTML]{FFB3B3}.063 & \cellcolor[HTML]{FFB3B3}.050 & \cellcolor[HTML]{FFFFB4}.032 & \cellcolor[HTML]{FFB3B3}.018 & \cellcolor[HTML]{FFD9B3}.026  & \cellcolor[HTML]{FFD9B3}.020 & \cellcolor[HTML]{FFD9B3}.034 & \cellcolor[HTML]{FFFFB4}.033 & \cellcolor[HTML]{FFD9B3}.065 & \cellcolor[HTML]{FFB3B3}.135 \\
\rowcolor[HTML]{FFFFFF} 
Ours, no neural                             & \cellcolor[HTML]{FFD9B3}.008 & .115                         & \cellcolor[HTML]{FFFFB4}.039 & \cellcolor[HTML]{FFFFB4}.058 & .071                         & .090                         & .053                         & \cellcolor[HTML]{FFD9B3}.026 & \cellcolor[HTML]{FFFFB4}.036  & .027                         & .045                         & .036                         & .070                         & .161                         \\
\cellcolor[HTML]{FFFFFF}Ours                & \cellcolor[HTML]{FFFFB4}.010 & \cellcolor[HTML]{FFD9B3}.104 & \cellcolor[HTML]{FFD9B3}.034 & \cellcolor[HTML]{FFB3B3}.046 & \cellcolor[HTML]{FFFFB4}.069 & \cellcolor[HTML]{FFD9B3}.055 & \cellcolor[HTML]{FFFFFF}.044 & \cellcolor[HTML]{FFFFB4}.024 & \cellcolor[HTML]{FFD9B3}.026  & \cellcolor[HTML]{FFFFB4}.022 & \cellcolor[HTML]{FFFFFF}.046 & \cellcolor[HTML]{FFFFFF}.044 & \cellcolor[HTML]{FFFFB4}.068 & \cellcolor[HTML]{FFFFB4}.149
\end{tabular}
}
\small{$^1$ requires object masks during training. ~~ $^2$ view synthesis method, not inverse rendering. ~~Red is best, followed by orange, then yellow. }
\caption{\textbf{LPIPS Results on the \emph{Shiny Blender} dataset from Ref-NeRF \cite{verbin2021ref} and \emph{Blender} dataset from NeRF \cite{mildenhall2021nerf}.}}
\label{tab:lpips}
\end{table*}

\begin{table*}[]
\resizebox{\linewidth}{!}{
\begin{tabular}{
>{\columncolor[HTML]{FFFFFF}}l |
>{\columncolor[HTML]{FFFFFF}}r 
>{\columncolor[HTML]{FFFFFF}}r 
>{\columncolor[HTML]{FFFFFF}}r 
>{\columncolor[HTML]{FFFFFF}}r 
>{\columncolor[HTML]{FFFFFF}}r 
>{\columncolor[HTML]{FFFFFF}}r |
>{\columncolor[HTML]{FFFFFF}}r 
>{\columncolor[HTML]{FFFFFF}}r 
>{\columncolor[HTML]{FFFFFF}}r 
>{\columncolor[HTML]{FFFFFF}}r 
>{\columncolor[HTML]{FFFFFF}}r 
>{\columncolor[HTML]{FFFFFF}}r 
>{\columncolor[HTML]{FFFFFF}}r 
>{\columncolor[HTML]{FFFFFF}}r }
$\text{MAE}^\circ \downarrow$ & \multicolumn{1}{l}{\cellcolor[HTML]{FFFFFF}teapot} & \multicolumn{1}{l}{\cellcolor[HTML]{FFFFFF}toaster} & \multicolumn{1}{l}{\cellcolor[HTML]{FFFFFF}car} & \multicolumn{1}{l}{\cellcolor[HTML]{FFFFFF}ball} & \multicolumn{1}{l}{\cellcolor[HTML]{FFFFFF}coffee} & \multicolumn{1}{l|}{\cellcolor[HTML]{FFFFFF}helmet} & \multicolumn{1}{l}{\cellcolor[HTML]{FFFFFF}chair} & \multicolumn{1}{l}{\cellcolor[HTML]{FFFFFF}lego} & \multicolumn{1}{l}{\cellcolor[HTML]{FFFFFF}materials} & \multicolumn{1}{l}{\cellcolor[HTML]{FFFFFF}mic} & \multicolumn{1}{l}{\cellcolor[HTML]{FFFFFF}hotdog} & \multicolumn{1}{l}{\cellcolor[HTML]{FFFFFF}ficus} & \multicolumn{1}{l}{\cellcolor[HTML]{FFFFFF}drums} & \multicolumn{1}{l}{\cellcolor[HTML]{FFFFFF}ship} \\ \hline
PhySG$^1$                     & 6.634                                              & \cellcolor[HTML]{FFFFB4}9.749                       & 8.844                                           & \cellcolor[HTML]{FFB3B3}0.700                    & 22.514                                             & \cellcolor[HTML]{FFB3B3}2.324                       & 18.569                                            & 40.244                                           & 18.986                                                & 26.053                                          & 28.572                                             & 35.974                                            & \cellcolor[HTML]{FFD9B3}21.696                    & 43.265                                           \\
NVDiffRec$^1$                 & \cellcolor[HTML]{FFB3B3}3.874                      & 14.336                                              & 15.286                                          & 5.584                                            & \cellcolor[HTML]{FFB3B3}11.132                     & 20.513                                              & 25.023                                            & 42.978                                           & 26.969                                                & 26.571                                          & 29.115                                             & 38.647                                            & 26.512                                            & 39.262                                           \\
NVDiffRecMC$^1$               & 5.928                                              & 11.905                                              & \cellcolor[HTML]{FFFFB4}8.357                   & 1.313                                            & 18.385                                             & 8.131                                               & 23.469                                            & 42.706                                           & 9.132                                                 & 26.184                                          & 26.470                                             & \cellcolor[HTML]{FFB3B3}34.324                    & 25.219                                            & 41.952                                           \\
Ref-NeRF$^2$                  & 9.234                                              & 42.870                                              & 14.927                                          & 1.548                                            & \cellcolor[HTML]{FFD9B3}12.240                     & 29.484                                              & 19.852                                            & \cellcolor[HTML]{FFB3B3}24.469                   & 9.531                                                 & 24.938                                          & \cellcolor[HTML]{FFFFB4}13.211                     & 41.052                                            & 27.853                                            & 31.707                                           \\
Ours, no integral image       & 10.078                                             & 39.779                                              & 28.744                                          & 45.998                                           & 14.776                                             & 28.550                                              & 20.594                                            & 26.712                                           & 27.462                                                & 29.956                                          & 15.188                                             & 36.543                                            & 32.118                                            & 37.464                                           \\
Ours, analytical derivative   & 6.400                                              & 21.403                                              & 10.685                                          & 21.145                                           & 15.425                                             & 8.800                                               & 17.801                                            & 26.852                                           & \cellcolor[HTML]{FFFFB4}8.960                         & \cellcolor[HTML]{FFB3B3}19.426                  & 14.138                                             & \cellcolor[HTML]{FFFFB4}35.505                    & 27.333                                            & 36.423                                           \\
Ours, single bounce           & 6.343                                              & \cellcolor[HTML]{FFD9B3}7.133                       & \cellcolor[HTML]{FFD9B3}7.746                   & \cellcolor[HTML]{FFFFB4}0.722                    & \cellcolor[HTML]{FFFFB4}12.950                     & \cellcolor[HTML]{FFFFB4}2.401                       & \cellcolor[HTML]{FFB3B3}14.285                    & \cellcolor[HTML]{FFFFB4}26.082                   & \cellcolor[HTML]{FFD9B3}8.315                         & \cellcolor[HTML]{FFD9B3}20.004                  & \cellcolor[HTML]{FFD9B3}10.263                     & 37.498                                            & \cellcolor[HTML]{FFFFB4}22.358                    & \cellcolor[HTML]{FFB3B3}29.771                   \\
Ours, no neural               & \cellcolor[HTML]{FFD9B3}4.508                      & 10.288                                              & 8.388                                           & \cellcolor[HTML]{FFD9B3}0.703                    & 14.745                                             & 5.320                                               & \cellcolor[HTML]{FFFFB4}17.503                    & 28.290                                           & 9.549                                                 & 20.181                                          & 13.356                                             & \cellcolor[HTML]{FFD9B3}35.298                    & 24.651                                            & \cellcolor[HTML]{FFFFB4}30.326                   \\
Ours                          & \cellcolor[HTML]{FFFFB4}5.672                      & \cellcolor[HTML]{FFB3B3}6.660                       & \cellcolor[HTML]{FFB3B3}7.742                   & 0.723                                            & 13.173                                             & \cellcolor[HTML]{FFD9B3}2.395                       & \cellcolor[HTML]{FFD9B3}14.330                    & \cellcolor[HTML]{FFD9B3}25.918                   & \cellcolor[HTML]{FFB3B3}8.101                         & \cellcolor[HTML]{FFFFB4}20.144                  & \cellcolor[HTML]{FFB3B3}10.043                     & 37.405                                            & \cellcolor[HTML]{FFB3B3}21.524                    & \cellcolor[HTML]{FFD9B3}30.152                  
\end{tabular}
}
\small{$^1$ requires object masks during training. ~~ $^2$ view synthesis method, not inverse rendering. ~~Red is best, followed by orange, then yellow. }
\caption{\textbf{MAE Results on the \emph{Shiny Blender} dataset from Ref-NeRF \cite{verbin2021ref} and \emph{Blender} dataset from NeRF \cite{mildenhall2021nerf}.}}
\label{tab:maes}
\end{table*}

\plotscene{ball} 
% See \cref{fig:ball} for the ball scene.

% \plotsceneshort{car} 
% See \cref{fig:car} for the car scene.

% \plotscene{helmet} 
% See \cref{fig:helmet} for the helmet scene.

\plotscene{coffee}

\plotscenetop{teapot} 
%See \cref{fig:teapot} for the teapot scene.

\plotscene{toaster} 
%See \cref{fig:toaster} for the toaster scene.

\plotsceneshort{materials} 
%See \cref{fig:materials} for the materials scene.

\plotscene{drums} 
%See \cref{fig:drums} for the drums scene.

\plotscene{ficus} 
%See \cref{fig:ficus} for the ficus scene.

\plotscenebottom{hotdog} 
%See \cref{fig:hotdog} for the hotdog scene.

\plotscene{mic} 
%See \cref{fig:mic} for the mic scene.

\plotship{ship}
%See \cref{fig:ship} for the ship scene.

\plotscene{lego} 
%See \cref{fig:lego} for the ship scene.