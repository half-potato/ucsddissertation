

\section{Preliminaries}

Our method combines aspects of \emph{volumetric} and \emph{surface-based} rendering; we begin with a brief introduction to each before describing our method in Section~\ref{sec:method}.

\subsection{Volume Rendering}

The core idea in emission-absorption volume rendering is that light accumulates along rays, with ``particles'' along the ray both emitting and absorbing light. The color measured by a camera pixel corresponding to a ray with origin $\point_c$ and direction $-\omegao$ is:
%Mathematically, volume rendering can be described as follows:
\begin{align} \label{eqn:volrendering2}
    % T(t) &= \exp\left( -\int_0^{t}\density(\point_c - t'\omegao)dt' \right) \\
    % L(\point_c, \omegao) &= \int_0^\infty T(t) \density(\point_c - t\omegao)L_o(\point_c-t\omegao, \omegao) dt,
    L(\point_c, \omegao) &= \int_0^\infty T(t) \density(\ray(t))L_o(\ray(t), \omegao) dt, \\
    \text{where}\;\; T(t) &= \exp\left( -\int_0^{t}\density(\ray(t'))dt' \right),    
    % T(t') &= \exp\left( -\int_0^{t'}\sigma_t(\point_c - t'\omegao)dt' \right) \\
\end{align}
where $\ray(t)=\point_c - t\omegao$ is a camera ray, $\density(\point)$ is the density at point $\point$ in the volume, $T$ denotes transmittance along the ray, and $L_o$ is the outgoing radiance. This formula is often approximated numerically using quadrature, following~\cite{max1995}:
% \begin{align}
% \label{eqn:volrenderingquadrature}
% % T_i &= \exp\left(-\sum_{j=0}^{i-1}\sigma_j\delta_j \right) \\
% % L(\point_c, \omegao) &= \sum_{i=0}^{N-1} T_i \left(1-\exp(-\sigma_i\delta_i)\right) \textbf{c}_i
% T_j &= \exp\left(-\sum_{k=0}^{j-1}\sigma_k\delta_k \right) \\
% % w_j &= T_j \left(1-\exp(-\sigma_j\delta_j)\right) \\
% L(\point_c, \omegao) &= \sum_{j=0}^{N-1} T_j \left(1-\exp(-\sigma_j\delta_j)\right) \textbf{c}_j
% \end{align}
\begin{align}
L(\point_c, \omegao) &\approx \sum_{j=0}^{N-1} w_j L_o(\ray(t_j), \omegao), \\
\label{eqn:volrenderingquadrature}
\text{where}\;\; w_j &= T_j \big(1-\exp(-\sigma(\ray(t_j))(t_{j+1}-t_j))\big), \\
\text{and}\;\; T_j &= \exp\left(-\sum_{k=0}^{j-1}\sigma(\ray(t_k))(t_{k+1}-t_k) \right).
% , ~~~ \delta_k = t_{k+1}-t_k,\\
% w_j &= T_j \left(1-\exp(-\sigma_j\delta_j)\right) \\
% L(\point_c, \omegao) &= \sum_{j=0}^{N-1} T_j \big(1-e^{-\sigma(\ray(t_j))\delta_j}\big) L_o(\ray(t_j), \omegao),
\end{align}
%using the same notation as in the continuous formula.
% where sample $j$ along the ray has optical density $\sigma_j$ and outgoing radiance $\textbf{c}_j$, and $\delta_j$ denotes the distance to the next sample. 
In this volume rendering paradigm, multiple 3D points can contribute to the color of a ray, with nearer and denser points contributing most. 




\subsection{Surface Rendering}

In surface rendering, and assuming fully-opaque surfaces, the color of a ray is determined solely by the light reflected by the first surface it encounters. Consider that the ray $\ray$ from camera position $\point_c$ in direction $-\omegao$ intersects its first surface at a 3D position $\point$. The ray color is then: 
\begin{equation}
    \label{eqn:rendering}
    L(\point_c, \omegao) = \int_{\mathbb{S}^2} f(\point, \omegao, \omegai)L_i(\point, \omegai) (\uvn(\point)\cdot\omegai)^+ d\omegai,
\end{equation}
\begin{equation}
    L(\point_c, \omegao) \approx \int_{\mathbb{S}^2} f(\point, \omegao, \omegai)d\omegai \int_{\mathbb{S}^2}L_i(\point, \omegai) (\uvn(\point)\cdot\omegai)^+ d\omegai
\end{equation}
where $\omegai$ is the direction of incident light, $\uvn(\point)$ is the surface normal at $\point$, $f$ is the \bsdf{} describing the material of the surface at $\point$, $L_i$ is the incident radiance, and $(\uvn(\point)\cdot\omegai)^+$ is a truncated cosine lobe (\ie its negative values are clipped to zero) facing outward from the surface. Note that this equation is recursive: $L_i$ inside the integral may be the outgoing radiance $L_o$ coming from a different scene point.
%that came directly from a source in the environment or bounced off another object.
The integral in Equation~\ref{eqn:rendering} is also typically approximated by discrete (and often random) sampling, and is the subject of a rich body of work \cite{cook1982reflectance, veach1998robust}.



