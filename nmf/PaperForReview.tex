\documentclass[10pt,twocolumn,letterpaper]{article}

\usepackage{iccv}
\usepackage{times}
\usepackage{epsfig}
\usepackage{graphicx}
\usepackage{amsmath}
\usepackage{amssymb}
\usepackage{caption}
\usepackage[accsupp]{axessibility}
\usepackage[table,xcdraw]{xcolor}

% Include other packages here, before hyperref.

% If you comment hyperref and then uncomment it, you should delete
% egpaper.aux before re-running latex.  (Or just hit 'q' on the first latex
% run, let it finish, and you should be clear).
\usepackage[pagebackref=true,breaklinks=true,letterpaper=true,colorlinks,bookmarks=false]{hyperref}

\iccvfinalcopy % *** Uncomment this line for the final submission

\def\iccvPaperID{3364} % *** Enter the ICCV Paper ID here
\def\httilde{\mbox{\tt\raisebox{-.5ex}{\symbol{126}}}}

% Pages are numbered in submission mode, and unnumbered in camera-ready
\ificcvfinal\pagestyle{empty}\fi


% Include other packages here, before hyperref.
\usepackage{booktabs}
%\usepackage{algorithm}
%\usepackage[noend]{algpseudocode}
\usepackage[export]{adjustbox}
\usepackage{tikz}
\usetikzlibrary{spy,calc}
\usepackage{subcaption}
%\usepackage{tabularx}

\usepackage{mathtools}
\DeclarePairedDelimiter\ceil{\lceil}{\rceil}
\DeclarePairedDelimiter\floor{\lfloor}{\rfloor}

% Support for easy cross-referencing
\usepackage[capitalize]{cleveref}
\crefname{section}{Sec.}{Secs.}
\Crefname{section}{Section}{Sections}
\Crefname{table}{Table}{Tables}
\crefname{table}{Tab.}{Tabs.}

\definecolor{turquoise}{cmyk}{0.65,0,0.1,0.3}
\definecolor{purple}{rgb}{0.65,0,0.65}
\definecolor{darkgreen}{rgb}{0, 0.5, 0}
\definecolor{orange}{rgb}{0.8, 0.6, 0.2}
\definecolor{darkred}{rgb}{0.6, 0.1, 0.05}
\definecolor{blueish}{rgb}{0.0, 0.3, .6}
\definecolor{lightgray}{rgb}{0.7, 0.7, .7}
\definecolor{pink}{rgb}{1, 0, 1}
\definecolor{greyblue}{rgb}{0.25, 0.25, 1}

\newcommand{\sfk}[1]{\textcolor{greyblue}{[SFK: #1]}}
\newcommand{\dv}[1]{\textcolor{purple}{[DV: #1]}}
\newcommand{\alex}[1]{\textcolor{darkgreen}{[AM: #1]}}
\newcommand{\todo}[1]{\textcolor{darkred}{[TODO: #1]}}
\newcommand{\tofix}[1]{\textcolor{magenta}{#1}}

\newcommand{\omegab}{\hat{\boldsymbol{\omega}}}
\newcommand{\ub}{\hat{\boldsymbol{u}}}
\newcommand{\vb}{\hat{\boldsymbol{v}}}
\newcommand{\hb}{\hat{\boldsymbol{h}}}
\newcommand{\omegai}{\hat{\boldsymbol{\omega}}_i}
\newcommand{\omegao}{\hat{\boldsymbol{\omega}}_o}
\newcommand{\omegak}{\hat{\boldsymbol{\omega}}_k}
\newcommand{\omegan}{\hat{\boldsymbol{\omega}}_i^n}
\newcommand{\feat}{\boldsymbol{x}}
\newcommand{\bp}{\mathbf{p}}
\newcommand{\bx}{\mathbf{x}}
\newcommand{\uvn}{\hat{\mathbf{n}}}
\newcommand{\tint}{\mathbf{t}}
\newcommand{\density}{\sigma}
\newcommand{\ray}{\vec{\mathbf{r}}}
%\newcommand{\modelname}{Illuminerfi}
\newcommand{\modelname}{Neural Microfacet Fields}
\newcommand{\bsdf}{BRDF}
\newcommand{\point}{\bp}

\begin{document}

\title{\modelname{} for Inverse Rendering}


\author{Alexander Mai\\
{\normalsize UC San Diego} \\
% Institution1 address\\
{\tt\small atm008@ucsd.edu}
% For a paper whose authors are all at the same institution,
% omit the following lines up until the closing ``}''.
% Additional authors and addresses can be added with ``\and'',
% just like the second author.
% To save space, use either the email address or home page, not both
\and
Dor Verbin\\
{\normalsize Google Research}\\
% First line of institution2 address\\
{\tt\small dorverbin@google.com}
\and
Falko Kuester\\
{\normalsize UC San Diego}\\
% First line of institution2 address\\
{\tt\small fkuester@ucsd.edu}
\and
Sara Fridovich-Keil\\
{\normalsize UC Berkeley}\\
% First line of institution2 address\\
{\tt\small sfk@eecs.berkeley.edu}
}




\newcommand{\plotall}[1]{%
  \adjincludegraphics[trim={{0\width} {0\height} {0\width} {0\height}}, clip, height=0.22\linewidth]{#1}%
}

% \newcommand{\plotscene}[1]{
% \begin{figure*}[h]
% \centering
% \plotall{nvdiffrec/#1/final.png}
% \plotall{nvdiffrec/#1/normals.png}
% \plotall{nvdiffrec/#1/mapped_pano.png} \\
% \plotall{nvdiffrecmc/#1/final.png}
% \plotall{nvdiffrecmc/#1/normals.png}
% \plotall{nvdiffrecmc/#1/mapped_pano.png} \\
% \plotall{images/#1/final.png}
% \plotall{images/#1/normals.png}
% \plotall{images/#1/mapped_pano.png} \\
% \plotall{images/#1/gt_final.png}
% \plotall{images/#1/gt_normals.png}
% \plotall{images/#1/gt_pano.png}
% \caption{\MakeUppercase#1 scene. Top row shows our results; bottom row shows ground truth. First column shows a rendered novel view, second column shows normals, and third column shows environment map.}
% \label{fig:#1}
% \end{figure*}
% }

% This version does one method per row, so it takes more space
% \newcommand{\plotscene}[1]{
% \begin{figure*}[h]
% \begin{tabular}{cccc}
% \rotatebox{90}{~~~~~~~~~~~~NVDiffRec} & \plotall{nvdiffrec/#1/final.png}
% & \plotall{nvdiffrec/#1/normals.png}
% & \plotall{nvdiffrec/#1/mapped_pano.png} \\
% \rotatebox{90}{~~~~~~~~~NVDiffRecMC} & \plotall{nvdiffrecmc/#1/final.png}
% & \plotall{nvdiffrecmc/#1/normals.png}
% & \plotall{nvdiffrecmc/#1/mapped_pano.png} \\
% \rotatebox{90}{~~~~~~~~~~~~~~~~~~Ours} & \plotall{images/#1/final.png}
% & \plotall{images/#1/normals.png}
% & \plotall{images/#1/mapped_pano.png} \\
% \rotatebox{90}{~~~~~~~~~~~Ground Truth} & \plotall{images/#1/gt_final.png}
% & \plotall{images/#1/gt_normals.png}
% & \plotall{images/#1/gt_pano.png} \\
% & Novel View & Normals & Environment Map
% \end{tabular}
% \caption{\textbf{Results on the \emph{#1} scene}, compared to NVDiffRec \cite{munkberg2021nvdiffrec} and NVDiffRecMC \cite{hasselgren2022nvdiffrecmc}.}
% \label{fig:#1}
% \end{figure*}
% }

\newcommand{\plotfourtrim}[1]{%
  \adjincludegraphics[trim={{0\width} {0.02\height} {0\width} {0.02\height}}, clip, width=0.24\linewidth]{#1}%
}
\newcommand{\plotfour}[1]{%
  \adjincludegraphics[trim={{0\width} {0.0\height} {0\width} {0.0\height}}, clip, width=0.24\linewidth]{#1}%
}
% This version does one method per column so it uses less space
\newcommand{\plotscene}[1]{
\begin{figure*}[h]
\begin{tabular}{l@{~~}c@{}c@{}c@{}c@{}}
\rotatebox{90}{~~~~~~~~~~~Novel View} & \plotfourtrim{images/#1/gt_final.png} & \plotfourtrim{images/#1/final.png} & \plotfourtrim{nvdiffrec/#1/final.png} & \plotfourtrim{nvdiffrecmc/#1/final.png} \\
\rotatebox{90}{~~~~~~~~~~~~~~Normals} & \plotfourtrim{images/#1/gt_normals.png} & \plotfourtrim{images/#1/normals.png} & \plotfourtrim{nvdiffrec/#1/normals.png} & \plotfourtrim{nvdiffrecmc/#1/normals.png} \\
\rotatebox{90}{Environment} & \plotfour{images/#1/gt_pano.png} & \plotfour{images/#1/mapped_pano.png} & \plotfour{nvdiffrec/#1/mapped_pano.png} & \plotfour{nvdiffrecmc/#1/mapped_pano.png} \\
& Ground Truth & Ours & NVDiffRec & NVDiffRecMC
\end{tabular}
\caption{\textbf{Results on the \emph{#1} scene}, compared to NVDiffRec \cite{munkberg2021nvdiffrec} and NVDiffRecMC \cite{hasselgren2022nvdiffrecmc}.}
\label{fig:#1}
\end{figure*}
}


\newcommand{\plotfourtrimtop}[1]{%
  \adjincludegraphics[trim={{0\width} {0.05\height} {0\width} {0.2\height}}, clip, width=0.24\linewidth]{#1}%
}

% This version does one method per column so it uses less space
\newcommand{\plotship}[1]{
\begin{figure*}[h]
\begin{tabular}{l@{~~}c@{}c@{}c@{}c@{}}
\rotatebox{90}{~~~~~~~~~~~Novel View} & \plotfourtrimtop{images/#1/gt_final.png} & \plotfourtrimtop{images/#1/final.png} & \plotfourtrimtop{nvdiffrec/#1/final.png} & \plotfourtrimtop{nvdiffrecmc/#1/final.png} \\
\rotatebox{90}{~~~~~~~~~~~~~~Normals} & \plotfourtrimtop{images/#1/gt_normals.png} & \plotfourtrimtop{images/#1/normals.png} & \plotfourtrimtop{nvdiffrec/#1/normals.png} & \plotfourtrimtop{nvdiffrecmc/#1/normals.png} \\
\rotatebox{90}{Environment} & \plotfour{images/#1/gt_pano.png} & \plotfour{images/#1/mapped_pano.png} & \plotfour{nvdiffrec/#1/mapped_pano.png} & \plotfour{nvdiffrecmc/#1/mapped_pano.png} \\
& Ground Truth & Ours & NVDiffRec & NVDiffRecMC
\end{tabular}
\caption{\textbf{Results on the \emph{#1} scene}, compared to NVDiffRec \cite{munkberg2021nvdiffrec} and NVDiffRecMC \cite{hasselgren2022nvdiffrecmc}. Since our method, NVDiffRec, and NVDiffRecMC do not model refraction, they are not able to handle the water well.}
\label{fig:#1}
\end{figure*}
}

% This is for the scenes that are short and wide, to save more vertical space
\newcommand{\plotfourtrimmore}[1]{%
  \adjincludegraphics[trim={{0\width} {0.2\height} {0\width} {0.3\height}}, clip, width=0.24\linewidth]{#1}%
}
\newcommand{\plotsceneshort}[1]{
\begin{figure*}[h]
\begin{tabular}{l@{~~}c@{}c@{}c@{}c@{}}
\rotatebox{90}{~~~~Novel View} & \plotfourtrimmore{images/#1/gt_final.png} & \plotfourtrimmore{images/#1/final.png} & \plotfourtrimmore{nvdiffrec/#1/final.png} & \plotfourtrimmore{nvdiffrecmc/#1/final.png} \\
\rotatebox{90}{~~~~~Normals} & \plotfourtrimmore{images/#1/gt_normals.png} & \plotfourtrimmore{images/#1/normals.png} & \plotfourtrimmore{nvdiffrec/#1/normals.png} & \plotfourtrimmore{nvdiffrecmc/#1/normals.png} \\
\rotatebox{90}{Environment} & \plotfour{images/#1/gt_pano.png} & \plotfour{images/#1/mapped_pano.png} & \plotfour{nvdiffrec/#1/mapped_pano.png} & \plotfour{nvdiffrecmc/#1/mapped_pano.png} \\
& Ground Truth & Ours & NVDiffRec & NVDiffRecMC
\end{tabular}
\caption{\textbf{Results on the \emph{#1} scene}, compared to NVDiffRec \cite{munkberg2021nvdiffrec} and NVDiffRecMC \cite{hasselgren2022nvdiffrecmc}.}
\label{fig:#1}
\end{figure*}
}

\newcommand{\plotscenebottom}[1]{
\begin{figure*}[h]
\begin{tabular}{l@{~~}c@{}c@{}c@{}c@{}}
\rotatebox{90}{~~~~Novel View} & \plotfourtrimtop{images/#1/gt_final.png} & \plotfourtrimtop{images/#1/final.png} & \plotfourtrimtop{nvdiffrec/#1/final.png} & \plotfourtrimtop{nvdiffrecmc/#1/final.png} \\
\rotatebox{90}{~~~~~Normals} & \plotfourtrimtop{images/#1/gt_normals.png} & \plotfourtrimtop{images/#1/normals.png} & \plotfourtrimtop{nvdiffrec/#1/normals.png} & \plotfourtrimtop{nvdiffrecmc/#1/normals.png} \\
\rotatebox{90}{Environment} & \plotfour{images/#1/gt_pano.png} & \plotfour{images/#1/mapped_pano.png} & \plotfour{nvdiffrec/#1/mapped_pano.png} & \plotfour{nvdiffrecmc/#1/mapped_pano.png} \\
& Ground Truth & Ours & NVDiffRec & NVDiffRecMC
\end{tabular}
\caption{\textbf{Results on the \emph{#1} scene}, compared to NVDiffRec \cite{munkberg2021nvdiffrec} and NVDiffRecMC \cite{hasselgren2022nvdiffrecmc}.}
\label{fig:#1}
\end{figure*}
}

\newcommand{\plotfourtrimbottom}[1]{%
  \adjincludegraphics[trim={{0\width} {0.2\height} {0\width} {0.05\height}}, clip, width=0.24\linewidth]{#1}%
}
\newcommand{\plotscenetop}[1]{
\begin{figure*}[h]
\begin{tabular}{l@{~~}c@{}c@{}c@{}c@{}}
\rotatebox{90}{~~~~Novel View} & \plotfourtrimbottom{images/#1/gt_final.png} & \plotfourtrimbottom{images/#1/final.png} & \plotfourtrimbottom{nvdiffrec/#1/final.png} & \plotfourtrimbottom{nvdiffrecmc/#1/final.png} \\
\rotatebox{90}{~~~~~Normals} & \plotfourtrimbottom{images/#1/gt_normals.png} & \plotfourtrimbottom{images/#1/normals.png} & \plotfourtrimbottom{nvdiffrec/#1/normals.png} & \plotfourtrimbottom{nvdiffrecmc/#1/normals.png} \\
\rotatebox{90}{Environment} & \plotfour{images/#1/gt_pano.png} & \plotfour{images/#1/mapped_pano.png} & \plotfour{nvdiffrec/#1/mapped_pano.png} & \plotfour{nvdiffrecmc/#1/mapped_pano.png} \\
& Ground Truth & Ours & NVDiffRec & NVDiffRecMC
\end{tabular}
\caption{\textbf{Results on the \emph{#1} scene}, compared to NVDiffRec \cite{munkberg2021nvdiffrec} and NVDiffRecMC \cite{hasselgren2022nvdiffrecmc}.}
\label{fig:#1}
\end{figure*}
}




\newcommand{\plottraining}[1]{
\begin{figure*}[h]
\centering
\begin{tabular}{c@{~}c@{~}c@{~}c}
\plotfour{#1_overtime/im100.png}
& \plotfour{#1_overtime/im500.png}
& \plotfour{#1_overtime/im900.png}
& \plotfour{#1_overtime/im30k.png} \\
\plotfour{#1_overtime/pano100.png}
& \plotfour{#1_overtime/pano500.png}
& \plotfour{#1_overtime/pano900.png}
& \plotfour{#1_overtime/pano30k.png} \\
100 steps & 500 steps & 900 steps & 30000 steps
\end{tabular}
\caption{\textbf{Snapshots of the \emph{#1} scene during optimization.} Early in training the object geometry is cloudy and the environment map is uniform, but as training proceeds the object develops a sharp surface and the environment map converges.}
\label{fig:#1 over time}
\end{figure*}
}

\newcommand{\plotlefttrim}[1]{%
  \adjincludegraphics[trim={{0\width} {0.06\height} {0\width} {0.06\height}}, clip, width=0.8\textwidth]{#1}%
}
\newcommand{\plotleft}[1]{%
  \adjincludegraphics[trim={{0\width} {0.0\height} {0\width} {0.0\height}}, clip, width=0.8\textwidth]{#1}%
}
\newcommand{\plotrighttrim}[1]{%
  \adjincludegraphics[trim={{0.03\width} {0.05\height} {0\width} {0.138\height}}, clip, width=\textwidth]{#1}%
}
\newcommand{\plotenvswap}{
\begin{figure}[h]
% \begin{table}[h]
\centering
\begin{minipage}{0.39\linewidth}
\centering
\begin{tabular}{c@{~}c@{}}
\plotlefttrim{images/combined/tint.png} & \rotatebox{90}{~~~~~~~~~~$\downarrow$} \\
\emph{Materials} \bsdf{} & \\
\plotleft{images/helmet/mapped_pano.png} & \rotatebox{90}{~~~~~~$\downarrow$} \\
\emph{Helmet} Lighting \\
\end{tabular}
\end{minipage} 
%
\begin{minipage}{0.59\linewidth}
\centering
\begin{tabular}{c@{}}
\plotrighttrim{images/combined/im000.png} \\
Combined Rendering (a)
\end{tabular}
\end{minipage}
% \end{table}
\includegraphics[width=\linewidth]{images/car_toaster.png}
Combined Rendering (b)

\caption{\textbf{Rendering with different illumination.} (a) shows the geometry and \bsdf{} (shown integrated against uniform white lighting) recovered from the \emph{materials} scene, rendered with the environment map recovered from the \emph{helmet} scene. (b) shows the geometry and \bsdf{} recovered from the \emph{toaster} and \emph{car} scene composed under the environment map recovered from the \emph{toaster} scene.}
\label{fig:environment swap}
\end{figure}
}
% \newcommand{\plotenvswap}{
% \begin{figure}[h]
% \centering
% \begin{tabular}{c@{}c@{}}
% \begin{tabular}{c@{}c@{}}
% \plotlefttrim{images/combined/tint.png} & \rotatebox{90}{~~~~~~~~~~$\downarrow$} \\
% \emph{Materials} \bsdf{} & \\
% \plotleft{images/helmet/mapped_pano.png} & \rotatebox{90}{~~~~~~$\downarrow$} 
% \end{tabular}
% & \plotrighttrim{images/combined/im000.png} \\
% \emph{Helmet} Lighting & Combined Rendering
% \end{tabular}
% \caption{\textbf{Rendering with different illumination.} Here we show the geometry and materials optimized from the \emph{materials} scene, rendered with the environment map optimized from the \emph{helmet} scene.}
% \label{fig:environment swap}
% \end{figure}
% }

% https://tikz.net/zoom/
\newif\ifblackandwhitecycle
\gdef\patternnumber{0}

\pgfkeys{/tikz/.cd,
    zoombox paths/.style={
        draw=orange,
        very thick
    },
    black and white/.is choice,
    black and white/.default=static,
    black and white/static/.style={
        draw=white,
        zoombox paths/.append style={
            draw=white,
            postaction={
                draw=black,
                loosely dashed
            }
        }
    },
    black and white/static/.code={
        \gdef\patternnumber{1}
    },
    black and white/cycle/.code={
        \blackandwhitecycletrue
        \gdef\patternnumber{1}
    },
    black and white pattern/.is choice,
    black and white pattern/0/.style={},
    black and white pattern/1/.style={
            draw=white,
            postaction={
                draw=black,
                dash pattern=on 2pt off 2pt
            }
    },
    black and white pattern/2/.style={
            draw=white,
            postaction={
                draw=black,
                dash pattern=on 4pt off 4pt
            }
    },
    black and white pattern/3/.style={
            draw=white,
            postaction={
                draw=black,
                dash pattern=on 4pt off 4pt on 1pt off 4pt
            }
    },
    black and white pattern/4/.style={
            draw=white,
            postaction={
                draw=black,
                dash pattern=on 4pt off 2pt on 2 pt off 2pt on 2 pt off 2pt
            }
    },
    zoomboxarray inner gap/.initial=5pt,
    zoomboxarray columns/.initial=2,
    zoomboxarray rows/.initial=1,
    zoomboxarray heightmultiplier/.initial=0.5,
    subfigurename/.initial={},
    figurename/.initial={zoombox},
    zoomboxarray/.style={
        execute at begin picture={
            \begin{scope}[
                spy using outlines={%
                    zoombox paths,
                    width=\imagewidth / \pgfkeysvalueof{/tikz/zoomboxarray columns} - (\pgfkeysvalueof{/tikz/zoomboxarray columns} - 1) / \pgfkeysvalueof{/tikz/zoomboxarray columns} * \pgfkeysvalueof{/tikz/zoomboxarray inner gap} -\pgflinewidth,
                    height=\pgfkeysvalueof{/tikz/zoomboxarray heightmultiplier} * (\imageheight / \pgfkeysvalueof{/tikz/zoomboxarray rows} - (\pgfkeysvalueof{/tikz/zoomboxarray rows} - 1) / \pgfkeysvalueof{/tikz/zoomboxarray rows} * \pgfkeysvalueof{/tikz/zoomboxarray inner gap}-\pgflinewidth),
                    magnification=3,
                    every spy on node/.style={
                        zoombox paths
                    },
                    every spy in node/.style={
                        zoombox paths
                    }
                }
            ]
        },
        execute at end picture={
            \end{scope}
     \gdef\patternnumber{0}
        },
        spymargin/.initial=0.5em,
        zoomboxes xshift/.initial=1,
        zoomboxes right/.code=\pgfkeys{/tikz/zoomboxes xshift=1},
        zoomboxes left/.code=\pgfkeys{/tikz/zoomboxes xshift=-1},
        zoomboxes yshift/.initial=0,
        zoomboxes above/.code={
            \pgfkeys{/tikz/zoomboxes yshift=1},
            \pgfkeys{/tikz/zoomboxes xshift=0}
        },
        zoomboxes below/.code={
            \pgfkeys{/tikz/zoomboxes yshift=-1},
            \pgfkeys{/tikz/zoomboxes xshift=0}
        },
        caption margin/.initial=0ex, %
    },
    adjust caption spacing/.code={},
    image container/.style={
        inner sep=0pt,
        at=(image.north),
        anchor=north,
        adjust caption spacing
    },
    zoomboxes container/.style={
        inner sep=0pt,
        at=(image.north),
        anchor=north,
        name=zoomboxes container,
        xshift=\pgfkeysvalueof{/tikz/zoomboxes xshift}*(\imagewidth+\pgfkeysvalueof{/tikz/spymargin}),
        yshift=\pgfkeysvalueof{/tikz/zoomboxes yshift}*(\imageheight+\pgfkeysvalueof{/tikz/spymargin}+\pgfkeysvalueof{/tikz/caption margin}),
        adjust caption spacing
    },
    calculate dimensions/.code={
        \pgfpointdiff{\pgfpointanchor{image}{south west} }{\pgfpointanchor{image}{north east} }
        \pgfgetlastxy{\imagewidth}{\imageheight}
        \global\let\imagewidth=\imagewidth
        \global\let\imageheight=\imageheight
        \gdef\columncount{1}
        \gdef\rowcount{1}
        \gdef\zoomboxcount{1}
    },
    image node/.style={
        inner sep=0pt,
        name=image,
        anchor=south west,
        append after command={
            [calculate dimensions]
            node [image container,subfigurename=\pgfkeysvalueof{/tikz/figurename}-image] {\phantomimage}
            node [zoomboxes container,subfigurename=\pgfkeysvalueof{/tikz/figurename}-zoom] {\phantomimage}
        }
    },
    color code/.style={
        zoombox paths/.append style={draw=#1}
    },
    connect zoomboxes/.style={
    spy connection path={\draw[draw=none,zoombox paths] (tikzspyonnode) -- (tikzspyinnode);}
    },
    help grid code/.code={
        \begin{scope}[
                x={(image.south east)},
                y={(image.north west)},
                font=\footnotesize,
                help lines,
                overlay
            ]
            \foreach \x in {0,1,...,9} {
                \draw(\x/10,0) -- (\x/10,1);
                \node [anchor=north] at (\x/10,0) {0.\x};
            }
            \foreach \y in {0,1,...,9} {
                \draw(0,\y/10) -- (1,\y/10);                        \node [anchor=east] at (0,\y/10) {0.\y};
            }
        \end{scope}
    },
    help grid/.style={
        append after command={
            [help grid code]
        }
    },
}

\newcommand\phantomimage{%
    \phantom{%
        \rule{\imagewidth}{\imageheight}%
    }%
}
\newcommand\zoombox[2][]{
    \begin{scope}[zoombox paths]
        \pgfmathsetmacro\xpos{
            (\columncount-1)*(\imagewidth / \pgfkeysvalueof{/tikz/zoomboxarray columns} + \pgfkeysvalueof{/tikz/zoomboxarray inner gap} / \pgfkeysvalueof{/tikz/zoomboxarray columns} ) + \pgflinewidth
        }
        \pgfmathsetmacro\ypos{
            (\rowcount-1) * (\imageheight / \pgfkeysvalueof{/tikz/zoomboxarray rows} + \pgfkeysvalueof{/tikz/zoomboxarray inner gap} / \pgfkeysvalueof{/tikz/zoomboxarray rows} ) + 0.5*\pgflinewidth
        }
        \edef\dospy{\noexpand\spy [
            #1,
            zoombox paths/.append style={
                black and white pattern=\patternnumber
            },
            every spy on node/.append style={#1},
            x=\imagewidth,
            y=\imageheight
        ] on (#2) in node [anchor=north west] at ($(zoomboxes container.north west)+(\xpos pt,-\ypos pt)$);}
        \dospy
        \pgfmathtruncatemacro\pgfmathresult{ifthenelse(\columncount==\pgfkeysvalueof{/tikz/zoomboxarray columns},\rowcount+1,\rowcount)}
        \global\let\rowcount=\pgfmathresult
        \pgfmathtruncatemacro\pgfmathresult{ifthenelse(\columncount==\pgfkeysvalueof{/tikz/zoomboxarray columns},1,\columncount+1)}
        \global\let\columncount=\pgfmathresult
        \ifblackandwhitecycle
            \pgfmathtruncatemacro{\newpatternnumber}{\patternnumber+1}
            \global\edef\patternnumber{\newpatternnumber}
        \fi
    \end{scope}
}





\newcommand{\plottrim}[1]{%
  \adjincludegraphics[trim={{0\width} {0.2\height} {0\width} {0.2\height}}, clip, width=0.24\linewidth]{#1}%
}

\definecolor{col1}{HTML}{e0d291}
\definecolor{col2}{HTML}{d66079}
\newcommand\plotzoomed[1]{
    \raisebox{-0.22\height}{
    \begin{tikzpicture}[
    zoomboxarray,
    zoomboxes below,
    connect zoomboxes,
    zoombox paths/.append style={thick}]
        \node[image node]{\plottrim{#1}};
        \zoombox[magnification=4,color code=col1]{0.28,0.670}
        \zoombox[magnification=4,color code=col2]{0.85,0.35}
        %
    \end{tikzpicture}
    }
}


\newcommand{\plotteaser}[1]{
\twocolumn[{
\renewcommand\twocolumn[1][]{#1}
\maketitle
\begin{center}
%\includegraphics[width=\linewidth,height=190px]{example-image-a}
\begin{tabular}{l@{~}c@{}c@{}c@{}} \vspace{-1.3cm}
\rotatebox{90}{~~~~~~~~~~~~~~~~Ours}
& \plotzoomed{images/#1/final.png}
& \plotzoomed{images/#1/normals.png}
& \plotall{images/#1/mapped_pano.png} \\ \vspace{-1cm}
\rotatebox{90}{~~~~~~~~~Ground Truth}
& \plotzoomed{images/#1/gt_final.png}
& \plotzoomed{images/#1/gt_normals.png}
& \plotall{images/#1/gt_pano.png} \\
& Novel View & Surface Normals & Environment Map
\end{tabular}
%\captionof{figure}{\modelname{} (top) recovers materials (\bsdf{}), geometry (density), and illumination (environment) from calibrated images of a scene, closely matching the ground truth (bottom). Here we show results on the \emph{#1} scene: a rendered novel view (left), density-based surface normals (middle), and environment map (right).}
\captionof{figure}{Our method (top) recovers materials, geometry, and illumination that closely resemble the ground truth (bottom), optimizing directly from calibrated images of a scene. Here we show results on the \emph{#1} scene from NeRF~\cite{mildenhall2021nerf}: a rendered novel view (left), surface normals (middle), and environment map illumination (right). Insets show high-fidelity reflections, including interreflections, as well as accurate geometry even in concave regions. \todo{bottom of our surface normals are getting cut off}}
\label{fig:#1_teaser}
\end{center}
}]
}


% version without zoom-in insets
% \newcommand{\plotteaser}[1]{
% \twocolumn[{
% \renewcommand\twocolumn[1][]{#1}
% \maketitle
% \begin{center}
% %\includegraphics[width=\linewidth,height=190px]{example-image-a}
% \begin{tabular}{cccc}
% \rotatebox{90}{\modelname{}}
% & \plotall{images/#1/final.png}
% & \plotall{images/#1/normals.png}
% & \plotall{images/#1/mapped_pano.png} \\
% \rotatebox{90}{~~~~~~~~~Ground Truth}
% & \plotall{images/#1/gt_final.png}
% & \plotall{images/#1/gt_normals.png}
% & \plotall{images/#1/gt_pano.png} \\
% & Novel View & Density-Based Surface Normals & Environment Map
% \end{tabular}
% %\captionof{figure}{\modelname{} (top) recovers materials (\bsdf{}), geometry (density), and illumination (environment) from calibrated images of a scene, closely matching the ground truth (bottom). Here we show results on the \emph{#1} scene: a rendered novel view (left), density-based surface normals (middle), and environment map (right).}
% \captionof{figure}{Our method (top) recovers materials, geometry, and illumination from calibrated images of a scene, that closely resemble the ground truth (bottom). Here we show results on the \emph{#1} scene from~\cite{mildenhall2021nerf}: a rendered novel view (left), surface normals (middle), and environment map illumination (right).}
% \label{fig:#1_teaser}
% \end{center}
% }]
% }
% \maketitle
%\plotteaser{materials} 
\twocolumn[{
\renewcommand\twocolumn[1][]{#1}
\maketitle
\begin{center}
\includegraphics*[width=\linewidth]{Illuminerfi-Fig1.pdf}
\captionof{figure}{Our method (top) recovers materials, geometry, and illumination that closely resemble the ground truth (bottom), optimizing directly from calibrated images of a scene. Here we show results on the \emph{materials} scene from NeRF~\cite{mildenhall2021nerf}: a rendered novel view (left), surface normals (middle), and environment map illumination (right). Insets show high-fidelity reflections, including interreflections, as well as accurate geometry, even in concave regions.}
\label{fig:materials_teaser}
\end{center}
}]
%%%%%%%%% ABSTRACT
\begin{abstract}


\vspace{-0.45cm}
\looseness=-1
We present \modelname{}, a method for recovering materials, geometry, and environment illumination from images of a scene. Our method uses a microfacet reflectance model within a volumetric setting by treating each sample along the ray as a (potentially non-opaque) surface.
Using surface-based Monte Carlo rendering in a volumetric setting enables our method to perform inverse rendering efficiently by combining decades of research in surface-based light transport with recent advances in volume rendering for view synthesis.
Our approach outperforms prior work in inverse rendering, capturing high fidelity geometry and high frequency illumination details; its novel view synthesis results are on par with state-of-the-art methods that do not recover illumination or materials.
\end{abstract}
\vspace{-0.2cm}
%%%%%%%%% BODY TEXT

\section{Introduction}

%Given a set of images of a scene and their respective camera poses we wish to estimate the 

Simultaneous recovery of the light sources illuminating a scene and the materials and geometry of objects inside it, given a collection of images, is a fundamental problem in computer vision and graphics. This decomposition enables editing and downstream usage of a scene: rendering it from novel viewpoints, and arbitrarily changing the scene's illumination, geometry, and material properties. This disentanglement is especially useful for creating 3D assets that can be inserted into other environments and realistically rendered under novel lighting conditions. 
%Disentangling these three highly coupled components---geometry, materials, and lighting---is necessary to generate reusable, relightable 3D assets, as well as to mix real and synthetic environments and objects.
%We achieve this decomposition by combining physics-based rendering with several recent advances in computer vision. 

%render images of the scene under novel viewpoints, lighting conditions, and with new materials. 

%We build our model upon recent advances in differential volumetric rendering, within which we add differentiable volumetric scattering using a surface-like phase function and Monte Carlo integration. By re-adding the volumetric scattering term to the radiative transfer equation, we are able to decompose the view dependent appearance of points in the scene into the material properties and the lighting. 

%Given calibrated images of a scene, our goal is to recover accurate surface geometry, material properties, and environment map. 


%Recent advances in volumetric rendering have been successful in decomposing the geometry of a scene from its appearance. However, while volumetric methods recover accurate geometry, they do not allow manipulating materials or illumination. We build our model upon these volumetric rendering methods, and use insights from differentiable surface rendering using path tracing for efficient rendering. This hybrid volume-surface representations allows 


%Many recent methods for novel view synthesis make use of \emph{volumetric rendering}, in which space is allowed to be foggy rather than forced either empty or solid. Although in the end most scenes do converge to solid objects surrounded by empty space, volumetric rendering makes for much easier optimization. This is in contrast to signed distance functions (SDFs), which model a surface directly; a surprising consequence of our work is that high-fidelity \emph{surfaces} can be recovered by modeling \emph{density}.

Recent methods for novel view synthesis based on neural radiance fields~\cite{mildenhall2021nerf} have been highly successful at decomposing scenes into their geometry and appearance components, enabling rendering from new, unobserved viewpoints.
%Recent advances in volumetric rendering have been successful in synthesizing novel views of a scene. 
However, the geometry and appearance recovered are often of limited use in manipulating either materials or illumination, since they model each point as a direction-dependent emitter rather than as reflecting the incident illumination.
To tackle the task of further decomposing appearance into illumination and materials, we return to a physical model of light-material interaction, which models a surface as a distribution of microfacets that \emph{reflect} light rather than emitting it. By explicitly modeling this interaction during optimization, our method can recover both material properties and the scene's illumination.






\looseness=-1
Our method uses a Monte Carlo rendering approach with a hybrid surface-volume representation, where the scene is parameterized as a 3D field of microfacets: the scene's geometry is represented as a volume density, but its materials are parameterized using a spatially varying Bidirectional Reflectance Distribution Function (BRDF). 
The volumetric representation of geometry has been shown to be effective for optimization~\cite{mildenhall2021nerf,yariv2021volume}, and treating each point in space as a microfaceted surface allows us to use ideas stemming from decades of prior work on material parameterization and efficient surface-based rendering. 
% When our method is applied to opaque objects, w
Despite its volumetric parameterization, we verify experimentally that our model shrinks into a surface around opaque objects, with all contributions to the color of a ray coming from the vicinity of its intersection with the object.




To summarize, our method (1) combines aspects of volume-based and surface-based rendering for effective optimization, enabling reconstructing high-fidelity scene geometry, materials, and lighting from a set of calibrated images; (2) uses an optimizable microfacet material model rendered using Monte Carlo integration with multi-bounce raytracing, allowing for realistic interreflections on nonconvex objects; and (3) is efficient: it optimizes a scene from scratch in $\sim$3 hours on a single NVIDIA GTX 3090.





\section{Related work}

Our work lies in the rich field of inverse rendering, in which the goal is to reconstruct the geometry, material properties, and illumination that gave rise to a set of observed images. This task is a severely underconstrained inverse problem, with challenges ranging from lack of differentiability~\cite{tzumao} to the computational cost-variance tradeoff of the forward rendering process~\cite{hasselgren2022nvdiffrecmc}.



Recent progress in inverse rendering, and in particular in view synthesis, has been driven by modeling scenes as radiance fields \cite{mildenhall2021nerf, mipnerf, mipnerf360}, which can produce photorealistic models of a scene based on calibrated images.

% \paragraph{Radiance fields.}
% Radiance fields such as NeRF \cite{mildenhall2021nerf, mipnerf, mipnerf360} are capable of producing photorealistic models of a scene based on calibrated photographs. Recent research has yielded dramatic improvements in optimization time \cite{plenoxels, ingp, tensorf}, recovery of dynamic scenes \cite{neuralvolumes, dynerf, dnerf, tensor4d, mixvoxels, v4d, nerfplayer, hypernerf, hexplane, kplane}, and recovery of in-the-wild scenes with variable appearance \cite{nerfw, boss2021nerd, boss2021neural, samurai, blocknerf, kplane}. Three lines of work are most relevant to ours: (1) modeling higher-fidelity reflections, (2) disentangling illumination and material properties, and (3) improving model interpretability.

\paragraph{Inverse rendering.}
%Inverse rendering is the task of recovering some combination of geometry, lighting, and materials necessary to render a scene from images of that scene. 
Inverse rendering techniques can be categorized based on the combination of unknowns recovered and assumptions made. 
Common assumptions include far field illumination, isotropic BRDFs, and no interreflections or self-occlusions. Early work by Ramamoorthi and Hanrahan~\cite{ramamoorthi2001signal} handled unknown lighting, texture, and BRDF by using spherical harmonic representations of both BRDF and lighting, which allowed recovering materials and low frequency lighting components. More recent methods used differentiable rendering of known geometry, first through differentiable rasterization~\cite{loper2014opendr, Liu_2019_ICCV, chen2019learning} and later through differentiable ray tracing~\cite{tzumao, azinovic2019inverse, park2020seeing}. Later methods built on differentiable ray tracing, making use of Signed Distance Fields (SDFs) to also reconstruct geometry~\cite{zhang2021physg, munkberg2021nvdiffrec, hasselgren2022nvdiffrecmc}. 

Following the success of NeRF~\cite{mildenhall2021nerf}, volumetric rendering has emerged as a useful tool for inverse rendering.
% inverse rendering methods that rely on volumetric rendering, often to obtain geometry, have emerged. 
Some methods based on volume rendering assume known lighting and only recover geometry and materials~\cite{srinivasan2021nerv, bi2020neural, asthana2022neural}, while others solve for both lighting and materials, but assume known geometry~\cite{lyu2022neural, zhang2021nerfactor} or geometry without self-occlusions~\cite{boss2021neural, boss2021nerd}. Some methods simultaneously recover illumination, geometry, and materials, but assume that illumination comes from a single point light source~\cite{guo2020object,iron-2022}. However, to the best of our knowledge none of these existing volumetric inverse rendering methods are able to capture high frequency lighting (and appearance of specular objects) from just the input images themselves. 

An additional challenge is in handling multi-bounce illumination, or interreflections, in which light from a source bounces off multiple objects before reaching the camera. In this case, computational tradeoffs are unavoidable due to the exponential growth of rays with the number of such bounces. Park \etal~\cite{park2020seeing} model interreflections assuming known geometry, but do not model materials, which is equivalent to treating all objects as perfect mirrors. Other methods use neural networks to cache visibility fields~\cite{srinivasan2021nerv, zhang2021nerfactor} or radiance transfer fields~\cite{lyu2022neural, guo2020object}. Our method handles interreflections by casting additional rays through the scene, using efficient Monte Carlo sampling.
%Similar to NVDiffRecMC~\cite{hasselgren2022nvdiffrecmc}, our model computes it directly by tracing ray bounces, but with a volumutric scene representation rather than an SDF.


\paragraph{Volumetric view synthesis.}
We base our representation of geometry on recent advances in volumetric view synthesis, following NeRF~\cite{mildenhall2021nerf}. Specifically, we retain the idea of using differentiable volumetric rendering to model geometry, using a voxel-based representation of the underlying density field~\cite{plenoxels, ingp, tensorf}.%, which enable our smooth estimation of normal vectors via 3D convolution. 

Of particular relevance to our work are prior methods such as~\cite{verbin2021ref,ge2023refneus} that are specifically designed for high-fidelity appearance of glossy objects. In general, most radiance field models fail at rendering high-frequency appearance caused by reflections from shiny materials under natural illumination, instead rendering blurry appearance~\cite{zhang2020nerf++}. To enable our method to handle these highly specular materials, and to improve the normal vectors estimated by our method, which are key to the rendered appearance, we utilize regularizers from Ref-NeRF~\cite{verbin2021ref}.




\section{Preliminaries}

Our method combines aspects of \emph{volumetric} and \emph{surface-based} rendering; we begin with a brief introduction to each before describing our method in Section~\ref{sec:method}.

\subsection{Volume Rendering}

The core idea in emission-absorption volume rendering is that light accumulates along rays, with ``particles'' along the ray both emitting and absorbing light. The color measured by a camera pixel corresponding to a ray with origin $\point_c$ and direction $-\omegao$ is:
%Mathematically, volume rendering can be described as follows:
\begin{align} \label{eqn:volrendering2}
    % T(t) &= \exp\left( -\int_0^{t}\density(\point_c - t'\omegao)dt' \right) \\
    % L(\point_c, \omegao) &= \int_0^\infty T(t) \density(\point_c - t\omegao)L_o(\point_c-t\omegao, \omegao) dt,
    L(\point_c, \omegao) &= \int_0^\infty T(t) \density(\ray(t))L_o(\ray(t), \omegao) dt, \\
    \text{where}\;\; T(t) &= \exp\left( -\int_0^{t}\density(\ray(t'))dt' \right),    
    % T(t') &= \exp\left( -\int_0^{t'}\sigma_t(\point_c - t'\omegao)dt' \right) \\
\end{align}
where $\ray(t)=\point_c - t\omegao$ is a camera ray, $\density(\point)$ is the density at point $\point$ in the volume, $T$ denotes transmittance along the ray, and $L_o$ is the outgoing radiance. This formula is often approximated numerically using quadrature, following~\cite{max1995}:
% \begin{align}
% \label{eqn:volrenderingquadrature}
% % T_i &= \exp\left(-\sum_{j=0}^{i-1}\sigma_j\delta_j \right) \\
% % L(\point_c, \omegao) &= \sum_{i=0}^{N-1} T_i \left(1-\exp(-\sigma_i\delta_i)\right) \textbf{c}_i
% T_j &= \exp\left(-\sum_{k=0}^{j-1}\sigma_k\delta_k \right) \\
% % w_j &= T_j \left(1-\exp(-\sigma_j\delta_j)\right) \\
% L(\point_c, \omegao) &= \sum_{j=0}^{N-1} T_j \left(1-\exp(-\sigma_j\delta_j)\right) \textbf{c}_j
% \end{align}
\begin{align}
L(\point_c, \omegao) &\approx \sum_{j=0}^{N-1} w_j L_o(\ray(t_j), \omegao), \\
\label{eqn:volrenderingquadrature}
\text{where}\;\; w_j &= T_j \big(1-\exp(-\sigma(\ray(t_j))(t_{j+1}-t_j))\big), \\
\text{and}\;\; T_j &= \exp\left(-\sum_{k=0}^{j-1}\sigma(\ray(t_k))(t_{k+1}-t_k) \right).
% , ~~~ \delta_k = t_{k+1}-t_k,\\
% w_j &= T_j \left(1-\exp(-\sigma_j\delta_j)\right) \\
% L(\point_c, \omegao) &= \sum_{j=0}^{N-1} T_j \big(1-e^{-\sigma(\ray(t_j))\delta_j}\big) L_o(\ray(t_j), \omegao),
\end{align}
%using the same notation as in the continuous formula.
% where sample $j$ along the ray has optical density $\sigma_j$ and outgoing radiance $\textbf{c}_j$, and $\delta_j$ denotes the distance to the next sample. 
In this volume rendering paradigm, multiple 3D points can contribute to the color of a ray, with nearer and denser points contributing most. 




\subsection{Surface Rendering}

In surface rendering, and assuming fully-opaque surfaces, the color of a ray is determined solely by the light reflected by the first surface it encounters. Consider that the ray $\ray$ from camera position $\point_c$ in direction $-\omegao$ intersects its first surface at a 3D position $\point$. The ray color is then: 
\begin{equation}
    \label{eqn:rendering}
    L(\point_c, \omegao) = \int_{\mathbb{S}^2} f(\point, \omegao, \omegai)L_i(\point, \omegai) (\uvn(\point)\cdot\omegai)^+ d\omegai,
\end{equation}
\begin{equation}
    L(\point_c, \omegao) \approx \int_{\mathbb{S}^2} f(\point, \omegao, \omegai)d\omegai \int_{\mathbb{S}^2}L_i(\point, \omegai) (\uvn(\point)\cdot\omegai)^+ d\omegai
\end{equation}
where $\omegai$ is the direction of incident light, $\uvn(\point)$ is the surface normal at $\point$, $f$ is the \bsdf{} describing the material of the surface at $\point$, $L_i$ is the incident radiance, and $(\uvn(\point)\cdot\omegai)^+$ is a truncated cosine lobe (\ie its negative values are clipped to zero) facing outward from the surface. Note that this equation is recursive: $L_i$ inside the integral may be the outgoing radiance $L_o$ coming from a different scene point.
%that came directly from a source in the environment or bounced off another object.
The integral in Equation~\ref{eqn:rendering} is also typically approximated by discrete (and often random) sampling, and is the subject of a rich body of work \cite{cook1982reflectance, veach1998robust}.




% \begin{figure*}[ht]
%     \centering
%     \includegraphics[width=\textwidth]{Illuminerfi-Fig2-V2-tall.pdf}
%     \caption{\textbf{We model the BRDF at each double sided micro surface as a convex combination of a diffuse lobe (top) and a specular lobe (bottom).} For the diffuse lobe, we (1a) integrate the radiance from the environment map against the cosine lobe of the normal vector $\uvn$ to get an irradiance map, then (1b) multiply this value by the albedo $\rho(\point)$ to get $L_d$. For the specular lobe, we start by (2a) sampling incoming light directions $\omegan$ according to the normal $\uvn$, view direction $\omegao$, and material roughness $\alpha$, then (2b) query the environment map and take a weighted mean with the neural BRDF term $h$ to get $L_s$. Finally, we (3) take a convex combination between these diffuse and specular components as the outgoing light $L_o$ at the sample point $\point$.}
%     \label{fig:methoddiagram}
% \end{figure*}

\begin{figure*}[ht]
    \centering
    \includegraphics[width=\textwidth]{Figure2.pdf}
    \caption{To render the color of a ray cast through the scene, we (a) evaluate density at each sample and compute each sample's volume rendering quadrature weight $w_i$, then (b) query the material properties and surface normal (flipped if it does not face the camera) at each sample point, which are used to (c) compute the color of each sample by using Monte Carlo integration of the surface rendering integral, where the number of samples used is proportional to the quadrature weight $w_i$. This sample color is then accumulated along the ray using the quadrature weight to get the final ray color.
    %\textbf{We model the BRDF at each double sided micro surface as a convex combination of a diffuse lobe (top) and a specular lobe (bottom).} For the diffuse lobe, we (1a) integrate the radiance from the environment map against the cosine lobe of the normal vector $\uvn$ to get an irradiance map, then (1b) multiply this value by the albedo $\rho(\point)$ to get $L_d$. For the specular lobe, we start by (2a) sampling incoming light directions $\omegan$ according to the normal $\uvn$, view direction $\omegao$, and material roughness $\alpha$, then (2b) query the environment map and take a weighted mean with the neural BRDF term $h$ to get $L_s$. Finally, we (3) take a convex combination between these diffuse and specular components as the outgoing light $L_o$ at the sample point $\point$.
    }
    \label{fig:methoddiagram}
\end{figure*}

\section{Method} \label{sec:method}

% In this paper, w
We present \modelname{} to tackle the problem of inverse rendering by combining volume and surface rendering, as shown in Figure~\ref{fig:methoddiagram}. Our method takes as input a collection of images ($100$ in our experiments) with known cameras, and outputs the volumetric density and normals, materials (\bsdf{}s), and far-field illumination (environment map) of the scene. We assume that all light sources are infinitely far away from the scene, though light may interact locally with multiple bounces through the scene.

\looseness=-1
In this section, we describe our representation of a scene and the rendering pipeline we use to map this representation into pixel values. 
Section~\ref{sec:main idea} introduces the main idea of our method, to build intuition before diving into the details. 
Section~\ref{section:geometry} describes our representation of the scene geometry, including density and normal vectors. Section~\ref{section:BRDF} describes our representation of materials and how they reflect light via the \bsdf{}.
Section~\ref{section:illumination} introduces our parameterization of illumination, which is based on a far-field environment map equipped with an efficient integrator for faster evaluations of the rendering integral. Finally, Section~\ref{section:rendering} describes the way we combine these different components to render pixel values in the scene.
%Section~\ref{section:optimization} describes the various ways we optimize this method to make the computation tractable.


\subsection{Main Idea} \label{sec:main idea}

The key to our method is a novel combination of the volume rendering and surface rendering paradigms: we model a density field as in volume rendering, and we model outgoing radiance at every point in space using surface-based light transport (approximated using Monte Carlo ray sampling).
%ray sampling as in surface rendering.
Volume rendering with a density field lends itself well to optimization: initializing geometry as a semi-transparent cloud creates useful gradients (see Figure~\ref{fig:toaster over time}), and allows for changes in geometry and topology. Using surface-based rendering allows modeling the interaction of light and materials, and enables recovering these materials.

We combine these paradigms by modeling a \emph{microfacet field}, in which each point in space is endowed with a volume density and a local micro-surface. Light accumulates along rays according to the volume rendering integral of Equation~\ref{eqn:volrendering2}, but the outgoing light of each 3D point is determined by surface rendering as in Equation~\ref{eqn:rendering}, using rays sampled according to its local micro-surface. 
This combination of volume-based and surface-based representation and rendering, shown in Figure~\ref{fig:methoddiagram}, enables us to optimize through a severely underconstrained inverse problem, recovering geometry, materials, and illumination simultaneously. 

\subsection{Geometry Parameterization} \label{section:geometry}

We represent geometry using a low-rank tensor data structure based on TensoRF~\cite{tensorf}, with small modifications described in 
% our supplement.
Appendix~\ref{appendix:optimization}. 
Our model stores both density $\density$ and a spatially-varying feature that is decoded into the material's \bsdf{} at every point in space. We initialize our model at low resolution and gradually upsample it during optimization (see Appendix~\ref{appendix:optimization} for details).

Similar to prior work~\cite{srinivasan2021nerv,verbin2021ref}, we use the negative normalized gradient of the density field as a field of ``volumetric normals.''
However, like~\cite{kuang2022neroic}, we found that numerically computing spatial gradients of the density field using finite differences rather than using analytic gradients leads to normal vectors that we can use directly, without using features predicted by a separate MLP. Additionally, these numerical gradients can be efficiently computed using 2D and 1D convolution using TensorRF's low-rank density decomposition (see Appendix~\ref{appendix:optimization}). These accurate normals are then used for rendering the appearance at a volumetric microfacet, as will be discussed in Sections~\ref{section:BRDF} and~\ref{section:rendering}. 

%The normalized gradient of the density field, or normal, is used in place of the surface normal to orient the phase function and give the volume surface like properties. 
%associated with position $\mathbf{x}$, $\hat{\mathbf{n}}(\mathbf{x})$, can be computed using the density field:
%\begin{equation}
%    \hat{\mathbf{n}}(\mathbf{x}) = \frac{\nabla \density(\mathbf{x})}{\|\nabla \density(\mathbf{x})\|_2}.
%\end{equation}


% Similar to Ref-NeRF~\cite{verbin2021ref}, we find that also outputting normals from our model significantly improves the performance of our system. 
Our volumetric normals are regularized using the orientation loss $\mathcal{R}_o$ introduced by Ref-NeRF~\cite{verbin2021ref}:
\begin{equation}
    \label{eqn:normal_penalty}
    \mathcal{R}_o = \sum_j w_j \max(0, -\uvn(\point_j) \cdot \omegao)^2,
    % + \max(0, \uvn' \cdot V)^2) \\
    % \mathcal{R}_p &= \sum_i w_i \|\uvn-\uvn'\|^2
\end{equation}
where $\omegao$ is the view direction facing towards the camera, and $\uvn(\point_j)$ is the normal vector at the $j$th point along the ray. The orientation loss $\mathcal{R}_o$ penalizes normals that face away from the camera yet contribute to the color of the ray (as quantified by weights $w_j$).

Because our volumetric normals are derived from the density field, this regularizer has a direct effect on the reconstructed geometry: it decreases the weight of backwards-facing normals by decreasing their density or increasing the density between them and the cameras, thereby promoting hard surfaces and improving reconstruction. 
Note that unlike Ref-NeRF, we do not use ``predicted normals'' for surface rendering, as our Gaussian-smoothed derivative filter achieves similar effect.
%we do not find it necessary to use ``predicted normals'' on top of the normals derived from our density field.


%\plottraining{toaster}
\begin{figure*}[ht]
    \centering
    \includegraphics[width=\textwidth]{Illuminerfi-Fig3.pdf}
    \caption{\textbf{Snapshots of the toaster scene during optimization.} The second row shows a cross section of the weights along each ray taken along the dotted line. Early in training the object geometry is cloudy and the environment map is uniform, but as training proceeds the object develops a sharp surface and the environment map converges.}
    \label{fig:toaster over time}
\end{figure*}


\subsection{Material Representation}  \label{section:BRDF}

We write our spatially varying \bsdf{} model $f$ %breaks light down into diffuse and specular lobes blended together using Schlick's approximation of the Fresnel term $F_r(\hb)$.
as a combination of diffuse and specular components:
\begin{align}
    % F_r(\hb) &= F_0(\point) + (1-F_0(\point))(1-\hb\cdot \omegao)^5 \\
    %f(\point, \omegao, \omegai) &= \nonumber \\
    %F_r(\hb) f_s&(\point, \omegao, \omegai) + (1-F_r(\hb)) \frac{\rho(\point)}{\pi}
    f(\omegao, \omegai) =
    \frac{\rho}{\pi}(1-F_r(\hb)) + F_r(\hb) f_s&(\omegao, \omegai),
\end{align}
where $\rho$ is the RGB albedo,
%at point $\point$, 
$\hb=\frac{\omegao+\omegai}{\|\omegao+\omegai\|}$ is the half vector,
$F_r(\hb)$ is the Fresnel term, $f_s(\omegao, \omegai)$ is the specular component of the \bsdf{}
%at point $\point$,
for outgoing view direction $\omegao$ and incident light direction $\omegai$. The spatial dependence of these terms on the point $\point$ is omitted for brevity. We use the Schlick approximation~\cite{schlick1994inexpensive} for the Fresnel term:
\begin{equation}
    \label{eqn:fresnel}
    F_r(\hb) = F_0(\point) + (1-F_0(\point))(1-\hb\cdot \omegao)^5,
\end{equation}
where $F_0(\point) \in [0, 1]^3$ is the spatially varying reflectance at the normal incidence at the point $\point$, and we base our specular \bsdf{} on the Cook-Torrance \bsdf~\cite{torrance1967theory}, using:
\begin{equation}
    \label{eqn:brdf}
    f_s(\omegao, \omegai) = \frac{D(\hb ; \alpha, \uvn, \omegao)G_1(\omegao, \hb)g(\omegai, \omegao)}{4\left(\uvn\cdot\omegao\right)\left(\uvn\cdot\omegai\right)
    },
\end{equation}
where $D$ is a Trowbridge-Reitz distribution~\cite{trowbridge1975average} (popularized by the GGX \bsdf{} model~\cite{walter2007microfacet}), $G_1$ is the Smith shadow masking function for the Trowbridge-Reitz distribution, and $g$ is a shallow multilayer perceptron (MLP) with a sigmoid nonlinearity at its output. 
The distribution $D$ models the roughness $\alpha$ of the material, 
%and provides an upper bound on the value of the \bsdf{} (which
and it is used for importance sampling, as described in Section~\ref{section:rendering}). The MLP $g$ captures other material properties not included in its explicit components.%, like the geometry shadowing term~\cite{heitz2015sggx}.

The parameters for each of the diffuse and specular \bsdf{} components are stored as features $\feat$ in the TensoRF representation (alongside density $\density$), allowing them to vary in space. We compute the roughness $\alpha$, albedo $\rho$ and reflectance at the normal incidence $F_0$ by applying a single linear layer with sigmoid activation to the spatially-localized features $\feat$. %, rather than the point coordinates $\point$ themselves.
Details about the architecture of the MLP $g$ and its input encoding can be found in Appendix~\ref{appendix:parameterization}.

We can approximately evaluate the rendering equation integral in Equation~\ref{eqn:rendering} more efficiently by assuming all microfacets at a point $\point$ have the same irradiance $E(\point)$:
\begin{align}
    % &E(\point) = \int_{\mathbb{S}^2} L_i(\point, \omegai) (\uvn(\point)\cdot\omegai)^+ d\omegai \\
    \int_{\mathbb{S}^2} &f(\point, \omegao, \omegai)L_i(\point, \omegai) (\uvn(\point)\cdot\omegai)^+ d\omegai  \\
    \label{eqn:approx_render}
    \approx \int_{\mathbb{S}^2} &F_r(\hb)f_s(\point, \omegao, \omegai)L_i(\point, \omegai) (\uvn(\point)\cdot\omegai)^+ d\omegai  \nonumber \\
    & + \frac{\rho(\point)}{\pi}E(\point) \int_{\mathbb{S}^2}(1-F_r(\hb))d\omegai .
\end{align}
The irradiance $E(\point)$, defined as:
\begin{equation}
    E(\point) = \int_{\mathbb{S}^2} L_i(\point, \omegai) (\uvn(\point)\cdot\omegai)^+ d\omegai ,
\end{equation}
can then be easily evaluated using an irradiance environment map approximated by low-degree spherical harmonics, as done by Ramamoorthi and Hanrahan~\cite{ramamoorthi2001efficient}. At every optimization step, we obtain the current irradiance environment map by integrating the environment map with spherical harmonic functions up to degree $2$, and combining the result with the coefficients of a clamped cosine lobe pointing in direction $\uvn(\point)$ to obtain the irradiance $E(\point)$~\cite{ramamoorthi2001efficient}.
Equation~\ref{eqn:approx_render} can then be importance sampled according to $D$ and integrated using Monte Carlo sampling of incoming light. 

We sample half vectors $\hb$ from the distribution of visible normals $D_{\omegao}$~\cite{heitz2018sampling}, which is defined as:
\begin{equation} \label{eqn:visible}
    D_{\omegao}(\hb) = \frac{G_1(\omegao, \hb)|\omegao\cdot\hb|D(\hb)}{|\omegao\cdot\uvn|} .
\end{equation}
However, to perform Monte Carlo integration of the rendering equation, we need to convert from half vector space to the space of incoming light, which requires multiplying by the determinant of the Jacobian of the reflection equation $\omegai = 2(\omegao\cdot \hb) \hb - \omegao$, which is $4 (\omegao\cdot\hb)$~\cite{walter2007microfacet}. Multiplying by the $\omegai\cdot\uvn$ 
term from Equation~\ref{eqn:rendering} as well as the Jacobian and Equation~\ref{eqn:visible} results in the following Monte Carlo estimate:
\begin{align}
    \label{eqn:montecarlo}
    L(\point, \omegao) \approx 
    \frac 1{N}\sum_{n=1}^{N} &F_r(\hat{\mathbf{h}}^n)g(\feat, \omegao, \omegan)L_i(\point, \omegan)\nonumber  \\
    & + (1-F_r(\hat{\mathbf{h}}^n)) \frac{\rho(\point)}{\pi}E(\point), \\
    \text{where }\hat{\mathbf{h}}^n \sim &D_{\omegao}(\,\,\cdot\,\,; \alpha(\point), \uvn(\point), \omegao), \nonumber \\
    \text{and }\omegan = &2(\omegao\cdot \hat{\mathbf{h}}^n) \hat{\mathbf{h}}^n - \omegao. \nonumber
\end{align}

% Our spatially varying \bsdf{} model $f$, is parameterized as a combination of a diffuse component and a neural microfacet specular component: 
% \begin{equation} \label{eqn:interpolate_f}
%     f(\point, \omegao, \omegai) = (1-\lambda(\point))\frac{1}{\pi}\rho(\point) + \lambda(\point) f_s(\point, \omegao, \omegai),
% \end{equation}
% where $\rho(\point)$ is the RGB albedo at point $\point$ and $f_s(\point, \omegao, \omegai)$ is the specular component of the \bsdf{} at point $\point$, for incident light direction $\omegai$ and outgoing view direction $\omegao$. Finally, $\lambda(\point) \in [0,1]$ interpolates between diffuse and specular materials in an energy preserving manner. 

% Our goal is to evaluate the rendering integral in Equation~\ref{eqn:rendering}. By linearity of integration, we can compute the diffuse contribution, $L_d$, and specular contribution, $L_s$, separately:
% \begin{align}
%     \label{eqn:interpolate_Lo}
%     L_o &= (1-\lambda(\point)) L_d + \lambda(\point) L_s, \\
%     \label{eqn:Ld}
%     L_d &= \rho(\point)\int_{\mathbb{S}^2} \frac{1}{\pi}L_i(\point, \omegai) (\uvn(\point)\cdot\omegai)^+ d\omegai, \\
%     \label{eqn:Ls}
%     L_s &= \int_{\mathbb{S}^2} f_s(\point, \omegao, \omegai)L_i(\point, \omegai) (\uvn(\point)\cdot\omegai)^+ d\omegai.
% \end{align}

% The diffuse component $L_d$ can be easily evaluated using an irradiance environment map approximated by low-degree spherical harmonics, as done by Ramamoorthi and Hanrahan~\cite{ramamoorthi2001efficient}. At every optimization step, we obtain the current irradiance environment map by integrating the environment map with spherical harmonic functions up to degree $2$, and combining the result with the coefficients of a clamped cosine lobe computed in~\cite{ramamoorthi2001efficient}.

% To evaluate $L_d$ for the sample points in the batch, we simply evaluate the spherical harmonics in the normal direction $\uvn(\point)$ and multiply channelwise by the RGB albedo, following Equation~\ref{eqn:Ld}.

% The specular component of the \bsdf{} is parameterized as a product of two components:
% \begin{equation}
%     % f_s(\point, \omegao, \omegai) = D(\alpha(\feat), \uvn(\point), \omegao)g(\feat, \omegai, \omegao),
%      f_s(\point, \omegao, \omegai) = \frac{D(\omegai ; \alpha(\point), \uvn(\point), \omegao)h(\point, \omegai, \omegao)}{\left(\uvn(\point)\cdot\omegai\right)^+}
% \end{equation}
% where $D$ is a Trowbridge-Reitz distribution~\cite{trowbridge1975average} (popularized by the GGX \bsdf{} model~\cite{walter2007microfacet}) and $h$ is a shallow multilayer perceptron (MLP) with a sigmoid nonlinearity at its output. 
% The distribution $D$ models the roughness $\alpha$ of the material, and provides an upper bound on the value of the \bsdf{} (which we also use for importance sampling, see Section~\ref{section:rendering}). The MLP $h$ captures other material properties, like the Fresnel effect.

% The parameters for each of the diffuse and specular \bsdf{} components are stored as features $\feat$ in the TensoRF representation (alongside density $\density$), allowing them to vary in space. We compute the roughness $\alpha$, albedo $\rho$, and weighting parameter $\lambda$ by applying a single linear layer with sigmoid activation to the spatially-localized features $\feat$, rather than the point coordinates $\point$ themselves. Details about the architecture of the MLP $h$ and its input encoding may be found in Appendix~\ref{appendix:parameterization}.

\subsection{Illumination} \label{section:illumination}

We model far field illumination using an environment map, represented using an equirectangular image with dimensions $H\times W\times 3$, with $H=512$ and $W=1024$.
% The raw values are stored in an $H\times W\times 3$ matrix of weights. 
%We achieve high dynamic range by applying an exponential function to the values before any downstream processing.
\looseness=-1
We map the optimizable parameters in our environment map parameterization into high dynamic range RGB values by applying an elementwise exponential function.

We use the term \emph{primary} to denote a ray originating at the camera, and \emph{secondary} to denote a ray bounced from a surface to evaluate its reflected light (whether that light arrives directly from the environment map, or from another scene point).

To minimize sampling noise, instead of using a single environment map element per secondary ray, we use the mean value over an axis-aligned rectangle in spherical coordinates, with the solid angle covered by the rectangle adjusted to match the sampling distribution $D$ at that point.  Concretely, to query the environment map at a given incident light direction, we first compute its corresponding spherical coordinates $(\theta, \phi)$, where $\theta$ and $\phi$ are the polar and azimuthal angles respectively. We then compute the mean value of the environment map over a (spherical) rectangle centered at $(\theta, \phi)$, whose size $\Delta\theta\times\Delta\phi$ we constrain to have aspect ratio $\frac{\Delta\theta}{\Delta\phi} = \sin\theta$. To choose the solid angle of the rectangle, $\Delta\theta\cdot\Delta\phi$, we modify the method from~\cite{colbert2007gpu}, which is based on Nyquist's sampling theorem (see Appendix~\ref{appendix:nyquist}). 
We compute these mean values efficiently using integral images, also known as summed-area tables~\cite{crow1984summed}. 





%There is also the matter of handling diffuse materials. Diffuse materials require more samples than shinier objects, but if we upper bound the diffuse color by the far field illumination it receives, we can compute the diffuse color very quickly using spherical harmonics~\cite{ramamoorthi2001efficient}. For every batch of rays, we integrate  samples from the environment map with spherical harmonics of degree 2, multiply by coefficients to multiply by the cosine term, then evaluate the spherical harmonics in the direction of the normal and multiply by the albedo to get the diffuse lighting. To ensure that our material conserves light, we linearly interpolate between the diffuse lighting and the specular lighting using the diffuse lighting brightness. 



% See \cref{fig:toaster over time} for the toaster scene during training.

\subsection{Rendering} \label{section:rendering}

\looseness=-1
In this section, we describe how the model components representing geometry (Section~\ref{section:geometry}), materials (Section~\ref{section:BRDF}), and illumination (Section~\ref{section:illumination}) are combined to render the color of a pixel. 
For each primary ray, we choose a set of sample points following the rejection-sampling strategy of~\cite{li2022nerfacc, ingp} to prioritize points near object surfaces. We query our geometry representation~\cite{tensorf} for the density $\sigma_j$ at each point, and use Equation~\ref{eqn:volrenderingquadrature} to estimate the contribution weight $w_j$ of each point to the final ray color.
%, where $\delta_i$ denotes the ray distance between adjacent samples:
% \begin{equation}
% w_i = \exp\left(-\sum_{j=0}^{i-1}\sigma_j\delta_j \right)\left(1-\exp(-\sigma_i\delta_i)\right)
% \end{equation}

% To compute the color for each of these points, we evaluate the spatially varying
% %feature $\feat_j$ and decode it into the
% albedo $\rho_j$, roughness $\alpha_j$, and reflect $\lambda_j$ (see Equation~\ref{eqn:interpolate_f}), as well as the normal vector $\uvn_j$.
% %each using a single linear layer and sigmoid activation.
% Using the albedo value, we evaluate the diffuse component $L_d$ as described in Section~\ref{section:BRDF}. We now describe how the specular component $L_s$ is estimated via Monte Carlo sampling.

To compute the color for each of these points, we compute the irradiance from the environment map and apply Equation~\ref{eqn:montecarlo} to obtain $L(\point_j, \omegao)$, as described in Section~\ref{section:BRDF}.

For each primary ray sample with weight $w_j$, we allocate $N=\floor*{w_j M}$ secondary rays, where $M=128$ is an upper bound on the total number of secondary rays for each primary ray (since $\sum_j w_j \leq 1$). The secondary rays are sampled according to the Trowbridge-Reitz distribution
%(the $D$ term of the specular \bsdf{} component),
using the normal vector $\uvn(\point_j)$ and roughness value $\alpha_j$ at the current sample.


% evaluate the spatially varying
% %feature $\feat_j$ and decode it into the
% albedo $\rho_j$, roughness $\alpha_j$, and reflect $\lambda_j$ (see Equation~\ref{eqn:interpolate_f}), as well as the normal vector $\uvn_j$.
% %each using a single linear layer and sigmoid activation.
% Using the albedo value, we evaluate the diffuse component $L_d$ as described in Section~\ref{section:BRDF}. We now describe how the specular component $L_s$ is estimated via Monte Carlo sampling.


% For each primary ray sample with weight $w_j$, we allocate $N=\floor*{w_j M}$ secondary rays for approximating specular appearance using the specular \bsdf{} component $f_s$. The secondary rays are sampled according to the Trowbridge-Reitz distribution (the $D$ term of the specular \bsdf{} component), using the normal vector $\uvn(\point_j)$ and roughness value $\alpha_j$ at the current sample,
% %, we draw $w_i M$ secondary rays from the \bsdf{} according to the Trowbridge-Reitz distribution~\cite{walter2007microfacet, heitz2018sampling}, 
% where $M=128$ is an upper bound on the total number of secondary rays for each primary ray (since $\sum_j w_j \leq 1$).
% %which is modified during optimization, as described later in this section.
% We pass these secondary rays, along with the spatially varying feature $\feat_j$, to the MLP component $h$ of the \bsdf{} to retrieve an RGB multiplier $h(\feat_j, \omegai, \omegao)$ for each incoming light value. 

%The contribution of each secondary ray to the final pixel color is then the product of $w_j$ and $h(\feat_j, \omegai, \omegao)$.
%We sort these secondary rays, add a small amount of randomness, and then select the $R$ secondary rays with the largest contributions to actually trace through the scene. 
When computing the incoming light $L_i(\point, \omegan)$, we save memory by randomly selecting a fixed number $R$ of secondary rays to interreflect through the scene while others index straight into the environment map.
%In order to limit the number of interreflecting rays, we must decide which secondary rays will be bounced again (up to $2$ bounces through the scene). 
We importance sample these $R$ secondary rays according to the largest channel of the weighted RGB multiplier $w_j\cdot g(\feat_j, \omegai, \omegao)$. In practice, we find that adding a small amount of random noise to the weighted RGB multiplier before choosing the $R$ largest values improves performance by slightly increasing the variation of the selected rays. 
%We also allow each ray to bounce at most two times through the scene. 

The remaining $N-R$ secondary rays with lower contribution are rendered more cheaply by evaluating the environment map directly rather than considering further interactions with the scene, as described in Section~\ref{section:illumination}.
%However, if a primary ray sample is allocated more than two secondary rays through the scene, any additional ``cheap'' secondary rays for that sample are culled rather than evaluated, mimicking Russian Roulette.

%For each primary sample $j$, we can now estimate its specular light component $L_s(\point_j, \omegao)$ using Equation~\ref{eqn:montecarlo}.
This combined sample color value $L(\point_j, \omegao)$ is then weighted by $w_j$, and the resulting colors are summed along the primary ray samples to produce pixel values.
Finally, the resulting pixel values are tonemapped to sRGB color and clipped to $[0, 1]$.

\paragraph{Dynamic batching.}
We apply the dynamic batch size strategy from NeRFAcc~\cite{li2022nerfacc} to the TensoRF~\cite{tensorf} sampler, which controls the number of samples per batch using the number of primary rays per batch. Since the number of secondary rays scales with the number of primary rays, we bound the maximal number of primary rays to avoid casting too many secondary rays. We use the same method to control $R$, the number of secondary bounces that retrace through the scene. During test time, we shuffle the image to match the training distribution, then unshuffle the image to get the result.

%As the scene optimizes and surfaces become opaque, more of the samples along each primary ray can be culled because they either have density zero or are occluded by regions of high density. As the scene's sparsity increases, we increase the number of primary rays per batch to keep the number of primary samples roughly constant throughout optimization. 
% In other words, during optimization we shift from batches of fewer primary rays with more samples per ray to batches of more primary rays with fewer and more targeted samples per ray. 
%However, we do not allow the number of primary rays per batch to exceed a maximum threshold, no matter how sparse the scene becomes.

%We use a similar adaptive batching strategy for secondary rays. For each primary ray, we allow up to $M$ secondary rays to be traced through the scene.
% (the allocation of these $M$ secondary rays is described in Section~\ref{section:rendering}). 
%We set $M$ inversely proportional to the batch size for primary rays, resulting in a roughly constant allocation of secondary rays to each primary ray sample (on average, we use two secondary rays per sample). However, once geometry becomes sufficiently sparse, our primary ray batch size hits its maximum allowed value. From this point onwards, the number of secondary rays $M$ per primary ray remains constant, even as the scene continues to sparsify and surfaces become sharp. This allows the number of secondary rays per sample to increase, reducing the noise in our estimated reflections as our surface quality improves. We follow a similar strategy to adaptively allocate tertiary rays, which likewise play a larger role later in optimization.


%In addition to far field illumination from the environment map, we also consider near field illumination by modeling ray bounces within the scene. 
%Each camera ray is cast through the scene and may bounce off zero, one, or two ``surfaces'' before exiting the scene and encountering the environment map. We control the number of secondary bounces using the same method used to control primary rays to manage memory constraints. During test time, we shuffle the image to match the training distribution, then unshuffle to image to get the result.

%Early in optimization, while surfaces are not yet formed, we approximate ray colors with a heavier reliance on the environment map; as surfaces solidify we devote greater computational budget towards simulating ray bounces for interreflections. 

%See Section~\ref{section:optimization} for greater detail on this computational strategy. By combining near and far field illumination in this way, we can approximately sample the incoming light $L(\point, \omegai)$ at the point $\point$ from direction $\omegai$.



We train using photometric loss and the normal penalty loss of Equation~\ref{eqn:normal_penalty}. 
Further details of our optimization and sampling methods can be found in Appendix~\ref{appendix:optimization}.
% the supplement.
\begin{tabular}{l|rrrrr}
                                     & \multicolumn{5}{c}{Mip-NeRF360}                                                                                                                                                                                             \\
                                     & \multicolumn{1}{l}{PSNR $\uparrow$} & \multicolumn{1}{l}{SSIM $\uparrow$} & \multicolumn{1}{l|}{LPIPS $\downarrow$}           & \multicolumn{1}{l}{GPU-hr $\downarrow$}             & \multicolumn{1}{l}{Mem. $\downarrow$} \\ \hline
3DGS~\cite{kerbl20233d}              & \cellcolor[HTML]{FFFFB4}27.48       & \cellcolor[HTML]{FFFFB4}.816        & \multicolumn{1}{r|}{.216}                         & {\color[HTML]{212121} 0.54}                         & 763MB                                 \\
StopThePop~\cite{radl2024stopthepop} & 27.33                               & \cellcolor[HTML]{FFFFB4}.816        & \multicolumn{1}{r|}{.212}                         & \cellcolor[HTML]{FFFFFF}{\color[HTML]{212121} 0.60} & 780MB                                 \\
3DGRT~\cite{moenne20243d}            & 27.20                               & \cellcolor[HTML]{FFD9B3}.818        & \multicolumn{1}{r|}{\cellcolor[HTML]{FFFFB4}.248} & $\approx$0.83                                       & 383MB                                 \\
SMERF~\cite{duckworth2024smerf}      & \cellcolor[HTML]{FFB3B3}27.99       & \cellcolor[HTML]{FFD9B3}.818        & \multicolumn{1}{r|}{\cellcolor[HTML]{FFD9B3}.238} & 272                                                 & 139MB                                 \\
Our model                            & \cellcolor[HTML]{FFD9B3}27.51       & \cellcolor[HTML]{FFB3B3}.825        & \multicolumn{1}{r|}{\cellcolor[HTML]{FFB3B3}.194} & \cellcolor[HTML]{FFFFFF}{\color[HTML]{212121} 1.04} & 1134MB                                \\ \hline
ZipNeRF~\cite{barron2023zip}         & 28.54                               & .828                                & \multicolumn{1}{r|}{.198}                         & \cellcolor[HTML]{FFFFFF}{\color[HTML]{212121} 32}   & -                                     \\ \hline
                                     & \multicolumn{5}{c}{Zip-NeRF}                                                                                                                                                                                                \\
                                     & \multicolumn{1}{l}{PSNR $\uparrow$} & \multicolumn{1}{l}{SSIM $\uparrow$} & \multicolumn{1}{l|}{LPIPS $\downarrow$}           & \multicolumn{1}{l}{GPU-hr $\downarrow$}             & \multicolumn{1}{l}{Mem. $\downarrow$} \\ \hline
3DGS~\cite{kerbl20233d}              & 25.84                               & .817                                & \multicolumn{1}{r|}{.358}                         & 1.07                                                & 222MB                                 \\
StopThePop~\cite{radl2024stopthepop} & \cellcolor[HTML]{FFFFB4}25.92       & \cellcolor[HTML]{FFFFB4}.819        & \multicolumn{1}{r|}{\cellcolor[HTML]{FFFFB4}.352} & 0.24                                                & 223MB                                 \\
3DGRT~\cite{moenne20243d}            & -                                   & -                                   & \multicolumn{1}{r|}{-}                            & -                                                   & \multicolumn{1}{l}{}                  \\
SMERF~\cite{duckworth2024smerf}      & \cellcolor[HTML]{FFB3B3}27.28       & \cellcolor[HTML]{FFD9B3}.829        & \multicolumn{1}{r|}{\cellcolor[HTML]{FFD9B3}.339} & 528                                                 & 4108MB                                \\
Our model                            & \cellcolor[HTML]{FFD9B3}26.58       & \cellcolor[HTML]{FFB3B3}.845        & \multicolumn{1}{r|}{\cellcolor[HTML]{FFB3B3}.308} & 1.26                                                & 1694MB                                \\ \hline
ZipNeRF~\cite{barron2023zip}         & 27.37                               & .836                                & \multicolumn{1}{r|}{.305}                         & 48                                                  & 662MB                                 \\ \hline
                                     & \multicolumn{5}{c}{Tanks\&Temples \& DeepBlending}                                                                                                                                                                          \\
                                     & \multicolumn{1}{l}{PSNR $\uparrow$} & \multicolumn{1}{l}{SSIM $\uparrow$} & \multicolumn{1}{l|}{LPIPS $\downarrow$}           & \multicolumn{1}{l}{GPU-hr $\downarrow$}             & \multicolumn{1}{l}{Mem. $\downarrow$} \\ \hline
3DGS~\cite{kerbl20233d}              & \cellcolor[HTML]{FFB3B3}26.65       & \cellcolor[HTML]{FFFFB4}.848        & \multicolumn{1}{r|}{.263}                         & 0.34                                                & 563MB                                 \\
StopThePop~\cite{radl2024stopthepop} & \cellcolor[HTML]{FFD9B3}26.60       & .847                                & \multicolumn{1}{r|}{\cellcolor[HTML]{FFFFB4}.252} & 0.39                                                & 549MB                                 \\
3DGRT~\cite{moenne20243d}            & 26.22                               & \cellcolor[HTML]{FFD9B3}.865        & \multicolumn{1}{r|}{\cellcolor[HTML]{FFD9B3}.254} & $\approx$0.83                                       & 388MB                                 \\
SMERF~\cite{duckworth2024smerf}      & -                                   & -                                   & \multicolumn{1}{r|}{-}                            & -                                                   & -                                     \\
Our model                            & \cellcolor[HTML]{FFFFB4}26.59       & \cellcolor[HTML]{FFB3B3}.889        & \multicolumn{1}{r|}{\cellcolor[HTML]{FFB3B3}.234} & 0.86                                                & 1173MB                                \\ \hline
ZipNeRF~\cite{barron2023zip}         & -                                   & -                                   & \multicolumn{1}{r|}{-}                            & -                                                   & \multicolumn{1}{l}{}                 
\end{tabular}


\section{Discussion and Limitations}

We introduced a novel and successful approach for the task of inverse rendering, using calibrated images alone to decompose a scene into its geometry, far-field illumination, and material properties. Our approach uses a combination of volumetric and surface-based rendering, in which we endow each point in space with both a density and a local microsurface, so that it can both occlude and reflect light from its environment. %Our approach combines the continuity of volume rendering, whose gradients are beneficial for optimization, with the Monte Carlo ray bouncing necessary to recover illumination. 
We verified experimentally that our method, which enjoys both the optimization landscape of volume rendering, as well as the richness and efficiency of surface-based Monte Carlo rendering, provides superior results relative to prior work. %We represent (1) geometry in the form of a density field, with normal vectors derived from that density field, (2) materials as a neural specular lobe, a diffuse albedo field, and a roughness field, and (3) illumination as an equirectangular HDR environment map.

However, our method is not without limitations. 
First, although it can handle non-convex geometry, it assumes far field illumination and thus performs poorly when this assumption is not satisfied. This issue is most clear in the \emph{coffee} scene in the Shiny Blender dataset, which has near-field light sources. It also does not handle interreflections very well, since the number of secondary bounces is limited, and due to our acceleration scheme of often directly querying the environment map, as explained in Section~\ref{section:rendering}.
Our model also does not handle refractive media,
%It also assumes reflection rather than refraction and thus fails to model refractive media,
which is most clear in the \emph{drums} and \emph{ship} scenes in the Blender dataset.
Our diffuse lighting model also assumes far field illumination, and thus fails to fully isolate shadows from the albedo, most obvious in the \emph{lego} scene in the Blender dataset. These scenes are visualized in Appendix~\ref{appendix:results}.
Another limitation of our model's \bsdf{} parameterization is that it struggles to represent anisotropic materials. 
Finally, our method exhibits some speckle noise in its renderings, particularly near bright spots, which may be alleviated by using a denoiser as was used in~\cite{hasselgren2022nvdiffrecmc}.
We believe these limitations would make for interesting future work, as well as applying our method to other field representations and larger scenes captured in the wild.
% , and multiple illumination reconstructions.

%Because the inverse rendering problem is underdetermined, our method must be guided by priors that bias it towards opaque surfaces, and nonetheless is susceptible to some artifacts. In particular, our method struggles with translucent materials like the drumheads in the \emph{drums} scene, and can only recover detailed illumination in the presence of specular materials. Finally, our method requires about 24GB of GPU memory, and about 3 hours to train on each scene, on a single GPU.
%Some of these limitations are inherent to the task of inverse rendering, but others may be improved by advances in 3D field representation and development of priors specialized to different types of scenes. 


\section{Acknowledgements}
We would like to thank Tzu-Mao Li for his advice, especially regarding the BRDF formulation, as well as his La Jolla renderer, which we used as a reference. AM is supported by ALERTCalifornia, developing technology to stay ahead of disasters, and the National Science Foundation under award \#CNS-1338192, MRI: Development of Advanced Visualization Instrumentation for the Collaborative Exploration of Big Data, and Kinsella Expedition Fund. This material is partially based upon work supported by the National Science Foundation under Award No. 2303178 to SFK.

%\clearpage

%%%%%%%%% REFERENCES
{\small
\bibliographystyle{ieee_fullname}
\bibliography{egbib}
}

\clearpage

\appendix
\section{Rectangle Size Derivation} \label{appendix:nyquist}

As mentioned in Section~\ref{section:illumination} of the main paper, a ray that reaches the environment map is assigned a color taken as the average color over an axis-aligned rectangle in spherical coordinates, where the shape of the rectangle depends on the ray's direction and the material's roughness at the ray's origin. %The area and aspect ratio of the sample rectangle are adapted to each ray. 
We modify the derivation of the area of the rectangle from GPU Gems~\cite{colbert2007gpu}. Let $N$ be the number of samples, $p(\omegai)$ be the probability density function of a given sample direction $\omegai$ for viewing direction $\omegao$, and let $H$ and $W$ be the height and width of the environment map (\ie its polar and azimuthal resolutions). The density $d(\omegai)$ of environment map pixels at a given direction must be inversely proportional to the Jacobian's determinant, $\sin\theta_i$, and it must also satisfy: 
\begin{equation}
    HW = \int_0^{2\pi}\int_0^\pi d(\omegai) \sin\theta_i d\theta_i d\phi_i,
\end{equation}
and therefore:
\begin{equation}
    d(\omegai) = \frac{HW}{2\pi^2\sin\theta_i}.
\end{equation}

The number of pixels per sample, which is the area of the rectangle, is then the total solid angle per sample, $Np(\omegai)$ multiplied by the number of pixels per solid angle:
\begin{equation}
    \Delta\theta\cdot \Delta\phi = \frac{Np(\omegai)}{d(\omegai)},
\end{equation}
where $\Delta\theta$ is the polar size of the rectangle, and $\Delta\phi$ is its azimuthal size, \ie the rectangle is $\Delta\theta\times\Delta\phi$, in equirectangular coordinates.
%We multiply the pixels per a solid angle, $1/d(\omegai)$, by the solid angle per sample, $N p(\omegai, \omegao)$, to get the pixels per a sample, which is our desired rectangle size.

As mentioned in Section~\ref{section:illumination} of the main paper, the aspect ratio of the rectangle is set to:
\begin{equation}
    \frac{\Delta\theta}{\Delta\phi} = \sin\theta_i,
\end{equation}
which yields:
\begin{align}
    \Delta\theta &= \sqrt{2\pi^2 \frac{N}{HW} p(\omegai)}\cdot \sin\theta_i ,\\
    \Delta\phi &= \sqrt{2\pi^2 \frac{N}{HW} p(\omegai)}.
\end{align}

%The ratio of width to height based on the distortion of the projection can then be used to calculate the width and height of the rectangle corresponding to each sample.

\section{BSDF Neural Network Parameterization} \label{appendix:parameterization}

Once we have sampled the incoming light directions $\omegai$ and their respective values $L(\point, \omegai)$, we transform them into the local shading frame to calculate the value of the neural shading network $h$. We parameterize the neural network with 2 hidden layers of width 64 as $h(\point, \omegao, \omegai, \uvn)$, where $\point$ is the position, $\omegao, \omegai$ are the outgoing and incoming light directions, respectively, and $\uvn$ is the normal. However, rather than feeding $\omegao$ and $\omegai$ to the network directly, we follow the schema laid out by Rusinkiewicz~\cite{rusinkiewicz1998new} and parameterize the input using the halfway vector $\boldsymbol{\hat{h}}$ and difference vector $\boldsymbol{\hat{d}}$ within the local shading frame $F(\uvn)$, which takes the world space to a frame of reference in which the normal vector points upwards:
\begin{align}
    T&={[0, 0, 1]}^\top \times \uvn \\
    F(\uvn) &= \begin{bmatrix}
    T, & \uvn\times T, & \uvn
    \end{bmatrix}^\top \\
    \boldsymbol{\hat{h}} &= F(\uvn) \frac{\omegai+\omegao}{\|\omegai+\omegao\|_2} \\
    \boldsymbol{\hat{d}} &= F(\boldsymbol{\hat{h}}) \omegai
\end{align}
where $\times$ is the cross product. Finally, we encode these two directions using spherical harmonics up to degree 4 (as done in Ref-NeRF~\cite{verbin2021ref} for encoding view directions), concatenate the feature vector $\feat$ from the field at point $\point$, and pass this as input to the network $h$. 





\section{Optimization and Architecture}\label{appendix:optimization}

To calculate the normal vectors of the density field, we apply a finite difference kernel, convolved with a $3\times 3$ Gaussian smoothing kernel with $\sigma=1$, then linearly interpolate between samples to get the resulting gradient in the 3D volume.
We supervise our method using photometric loss, along with the orientation loss of Equation~\ref{eqn:normal_penalty}. Like TensoRF, we use a learning rate of $0.02$ for the rank $1$ and $2$ tensor components, and a learning rate of $10^{-3}$ for everything else. We use Adam~\cite{kingma2014adam} with $\beta_1=0.9,\beta_2=0.99, \varepsilon=10^{-15}$. %We decay the learning rate log-linearly in the same way as Ref-NeRF~\cite{verbin2021ref}.
Similar to Ref-NeRF~\cite{verbin2021ref}, we use log-linear learning rate decay with a total decay of $d_w = 10^{-3}$ and a warmup of $N_w = 100$ steps and a decay multiplier of $m_w = 0.1$ over $N_T=3\cdot 10^4$ total iterations. This gives us the following formula for the learning rate multiplier for some iteration $i$:
\begin{equation}
    \left[m_w + (1-m_w) \sin\frac\pi2 \text{clip}\left(\frac{i}{N_w}, 0, 1\right)\right] e^{i/N_T}\log(d_w)
\end{equation}

We initialize the environment map to a constant value of $0.5$. 
Finally, we upsample the resolution of TensoRF from $32^3$ up to $300^3$ cube-root-linearly at steps $500, 1000, 2000, 3000, 4000, 5500, 7000$, and don't shrink the volume to fit the model.

To further reduce the variance of the estimated value of the rendering equation (see Equation~\ref{eqn:montecarlo}), we use quasi-random sampling sequences. Specifically, we use a Sobol sequence~\cite{sable1967} with Owens scrambling~\cite{owen1995randomly}, which gives the procedural sequence necessary for assigning an arbitrary number of secondary ray samples to each primary ray sample. We then apply Cranley-Patterson rotation~\cite{cranley1976randomization} to avoid needing to redraw samples.


\section{Additional Results}
\label{appendix:results}
Tables 2-5 contain full per-scene metrics for our method as well as ablations and baselines. Visual comparisons are also provided in Figures 7-17.

\begin{table*}[]
\resizebox{\linewidth}{!}{
\begin{tabular}{l|rrrrrr|rrrrrrrr}
\rowcolor[HTML]{FFFFFF} 
PSNR $\uparrow$                             & \multicolumn{1}{l}{\cellcolor[HTML]{FFFFFF}teapot} & \multicolumn{1}{l}{\cellcolor[HTML]{FFFFFF}toaster} & \multicolumn{1}{l}{\cellcolor[HTML]{FFFFFF}car} & \multicolumn{1}{l}{\cellcolor[HTML]{FFFFFF}ball} & \multicolumn{1}{l}{\cellcolor[HTML]{FFFFFF}coffee} & \multicolumn{1}{l|}{\cellcolor[HTML]{FFFFFF}helmet} & \multicolumn{1}{l}{\cellcolor[HTML]{FFFFFF}chair} & \multicolumn{1}{l}{\cellcolor[HTML]{FFFFFF}lego} & \multicolumn{1}{l}{\cellcolor[HTML]{FFFFFF}materials} & \multicolumn{1}{l}{\cellcolor[HTML]{FFFFFF}mic} & \multicolumn{1}{l}{\cellcolor[HTML]{FFFFFF}hotdog} & \multicolumn{1}{l}{\cellcolor[HTML]{FFFFFF}ficus} & \multicolumn{1}{l}{\cellcolor[HTML]{FFFFFF}drums} & \multicolumn{1}{l}{\cellcolor[HTML]{FFFFFF}ship} \\ \hline
\rowcolor[HTML]{FFFFFF} 
PhySG$^1$                                   & 35.83                                              & 18.59                                               & 24.40                                           & 27.24                                            & 23.71                                              & 27.51                                               & 21.87                                             & 17.10                                            & 18.02                                                 & 19.16                                           & 24.49                                              & 15.25                                             & 14.35                                             & 18.06                                            \\
\rowcolor[HTML]{FFFFFF} 
NVDiffRec$^1$                               & 40.13                                              & 24.10                                               & 27.13                                           & 30.77                                            & 30.58                                              & 26.66                                               & 32.03                                             & 29.07                                            & 25.03                                                 & 30.72                                           & 33.05                                              & \cellcolor[HTML]{FFD9B3}31.18                     & 24.53                                             & 24.68                                            \\
\rowcolor[HTML]{FFFFFF} 
NVDiffRecMC$^1$                             & 37.91                                              & 21.93                                               & 25.84                                           & 28.89                                            & 29.06                                              & 25.57                                               & 28.13                                             & 26.46                                            & 25.64                                                 & 29.03                                           & 30.56                                              & 25.32                                             & 22.78                                             & 18.59                                            \\
\rowcolor[HTML]{FFB3B3} 
\cellcolor[HTML]{FFFFFF}Ref-NeRF$^2$        & 47.90                                              & \cellcolor[HTML]{FFFFFF}25.70                       & 30.82                                           & 47.46                                            & 34.21                                              & \cellcolor[HTML]{FFFFB4}29.68                       & 35.83                                             & 36.25                                            & 35.41                                                 & 36.76                                           & 37.72                                              & 33.91                                             & 25.79                                             & 30.28                                            \\
\rowcolor[HTML]{FFFFFF} 
Ours, no integral image                     & 42.61                                              & 18.36                                               & 25.32                                           & 21.70                                            & 31.15                                              & 24.82                                               & 30.35                                             & 30.16                                            & 25.62                                                 & 30.03                                           & 33.34                                              & 28.44                                             & 24.04                                             & 25.78                                            \\
\rowcolor[HTML]{FFFFFF} 
Ours, analytical derivative                 & 43.57                                              & 21.57                                               & 27.72                                           & 22.75                                            & 31.08                                              & 28.61                                               & 30.49                                             & 30.23                                            & 28.70                                                 & 31.19                                           & 33.55                                              & 27.83                                             & 24.15                                             & 25.40                                            \\
\rowcolor[HTML]{FFFFB4} 
\cellcolor[HTML]{FFFFFF}Ours, single bounce & 45.23                                              & \cellcolor[HTML]{FFD9B3}26.91                       & 30.13                                           & 38.38                                            & 31.39                                              & \cellcolor[HTML]{FFD9B3}34.32                       & \cellcolor[HTML]{FFD9B3}32.57                     & 32.83                                            & 30.92                                                 & \cellcolor[HTML]{FFD9B3}32.49                   & 35.07                                              & 29.24                                             & \cellcolor[HTML]{FFD9B3}24.99                     & 27.32                                            \\
\rowcolor[HTML]{FFFFFF} 
Ours, no neural                             & 45.21                                              & \cellcolor[HTML]{FFFFB4}25.73                       & 29.03                                           & 37.41                                            & 30.99                                              & 29.63                                               & 30.62                                             & 31.00                                            & 29.37                                                 & 31.29                                           & 33.88                                              & 28.10                                             & 24.52                                             & 26.44                                            \\
\rowcolor[HTML]{FFD9B3} 
\cellcolor[HTML]{FFFFFF}Ours                & 45.29                                              & \cellcolor[HTML]{FFB3B3}27.52                       & 30.28                                           & 38.41                                            & 31.47                                              & \cellcolor[HTML]{FFB3B3}34.38                       & \cellcolor[HTML]{FFFFB4}32.27                     & 32.98                                            & 31.19                                                 & \cellcolor[HTML]{FFFFB4}32.41                   & 35.23                                              & \cellcolor[HTML]{FFFFFF}29.24                     & \cellcolor[HTML]{FFFFB4}24.96                     & 27.37                                           
\end{tabular}
}
\small{$^1$ requires object masks during training. ~~ $^2$ view synthesis method, not inverse rendering. ~~ Red is best, followed by orange, then yellow. }
\caption{\textbf{PSNR Results on the \emph{Shiny Blender} dataset from Ref-NeRF \cite{verbin2021ref} and \emph{Blender} dataset from NeRF \cite{mildenhall2021nerf}.}}
\label{tab:psnrs}
\end{table*}

\begin{table*}[]
\resizebox{\linewidth}{!}{
\begin{tabular}{l|rrrrrr|rrrrrrrr}
\rowcolor[HTML]{FFFFFF} 
SSIM $\uparrow$                             & \multicolumn{1}{l}{\cellcolor[HTML]{FFFFFF}teapot} & \multicolumn{1}{l}{\cellcolor[HTML]{FFFFFF}toaster} & \multicolumn{1}{l}{\cellcolor[HTML]{FFFFFF}car} & \multicolumn{1}{l}{\cellcolor[HTML]{FFFFFF}ball} & \multicolumn{1}{l}{\cellcolor[HTML]{FFFFFF}coffee} & \multicolumn{1}{l|}{\cellcolor[HTML]{FFFFFF}helmet} & \multicolumn{1}{l}{\cellcolor[HTML]{FFFFFF}chair} & \multicolumn{1}{l}{\cellcolor[HTML]{FFFFFF}lego} & \multicolumn{1}{l}{\cellcolor[HTML]{FFFFFF}materials} & \multicolumn{1}{l}{\cellcolor[HTML]{FFFFFF}mic} & \multicolumn{1}{l}{\cellcolor[HTML]{FFFFFF}hotdog} & \multicolumn{1}{l}{\cellcolor[HTML]{FFFFFF}ficus} & \multicolumn{1}{l}{\cellcolor[HTML]{FFFFFF}drums} & \multicolumn{1}{l}{\cellcolor[HTML]{FFFFFF}ship} \\ \hline
\rowcolor[HTML]{FFFFFF} 
PhySG$^1$                                   & .990                                               & .805                                                & .910                                            & .947                                             & .922                                               & .953                                                & .890                                              & .812                                             & .837                                                  & .904                                            & .894                                               & .861                                              & .823                                              & .756                                             \\
\rowcolor[HTML]{FFFFFF} 
NVDiffRec$^1$                               & .993                                               & .898                                                & .938                                            & .949                                             & .959                                               & .931                                                & \cellcolor[HTML]{FFD9B3}.969                      & .949                                             & .923                                                  & \cellcolor[HTML]{FFFFB4}.977                    & \cellcolor[HTML]{FFD9B3}.973                       & \cellcolor[HTML]{FFD9B3}.970                      & .916                                              & \cellcolor[HTML]{FFFFB4}.833                     \\
\rowcolor[HTML]{FFFFFF} 
NVDiffRecMC$^1$                             & .990                                               & .842                                                & .913                                            & .849                                             & .942                                               & .877                                                & .932                                              & .909                                             & .911                                                  & .961                                            & .945                                               & .937                                              & .906                                              & .732                                             \\
\rowcolor[HTML]{FFB3B3} 
\cellcolor[HTML]{FFFFFF}Ref-NeRF$^2$        & .998                                               & .922                                                & .955                                            & .995                                             & .974                                               & \cellcolor[HTML]{FFFFB4}.958                        & .984                                              & .981                                             & .983                                                  & .992                                            & .984                                               & .983                                              & .937                                              & .880                                             \\
\rowcolor[HTML]{FFFFFF} 
Ours, no integral image                     & .994                                               & .734                                                & .895                                            & .753                                             & .959                                               & .880                                                & .946                                              & .946                                             & .896                                                  & .962                                            & .954                                               & .953                                              & .905                                              & .794                                             \\
\rowcolor[HTML]{FFFFFF} 
Ours, analytical derivative                 & \cellcolor[HTML]{FFFFB4}.995                       & .798                                                & .925                                            & .790                                             & .959                                               & .930                                                & .948                                              & .943                                             & .936                                                  & .972                                            & .958                                               & .950                                              & .910                                              & .787                                             \\
\cellcolor[HTML]{FFFFFF}Ours, single bounce & \cellcolor[HTML]{FFD9B3}.996                       & \cellcolor[HTML]{FFFFB4}.909                        & \cellcolor[HTML]{FFD9B3}.951                    & \cellcolor[HTML]{FFD9B3}.983                     & \cellcolor[HTML]{FFD9B3}.962                       & \cellcolor[HTML]{FFB3B3}.971                        & \cellcolor[HTML]{FFFFB4}.964                      & \cellcolor[HTML]{FFD9B3}.966                     & \cellcolor[HTML]{FFFFB4}.957                          & \cellcolor[HTML]{FFD9B3}.978                    & \cellcolor[HTML]{FFFFB4}.969                       & \cellcolor[HTML]{FFFFB4}.959                      & \cellcolor[HTML]{FFD9B3}.922                      & \cellcolor[HTML]{FFD9B3}.835                     \\
\rowcolor[HTML]{FFFFFF} 
Ours, no neural                             & \cellcolor[HTML]{FFD9B3}.996                       & .903                                                & \cellcolor[HTML]{FFFFB4}.945                                            & \cellcolor[HTML]{FFFFB4}.980                                             & .959                                               & .947                                                & .949                                              & .952                                             & .945                                                  & .972                                            & .960                                               & .954                                              & .916                                              & .816                                             \\
\cellcolor[HTML]{FFFFFF}Ours                & \cellcolor[HTML]{FFD9B3}.996                       & \cellcolor[HTML]{FFD9B3}.917                        & \cellcolor[HTML]{FFD9B3}.951                    & \cellcolor[HTML]{FFD9B3}.983                     & \cellcolor[HTML]{FFFFB4}.960                       & \cellcolor[HTML]{FFD9B3}.969                        & \cellcolor[HTML]{FFFFFF}.956                      & \cellcolor[HTML]{FFFFB4}.963                     & \cellcolor[HTML]{FFD9B3}.959                          & \cellcolor[HTML]{FFFFFF}.977                    & \cellcolor[HTML]{FFFFFF}.964                       & \cellcolor[HTML]{FFFFFF}.952                      & \cellcolor[HTML]{FFFFB4}.917                      & \cellcolor[HTML]{FFFFFF}.828                    
\end{tabular}
}
\small{$^1$ requires object masks during training. ~~ $^2$ view synthesis method, not inverse rendering.~~ Red is best, followed by orange, then yellow. }
\caption{\textbf{SSIM Results on the \emph{Shiny Blender} dataset from Ref-NeRF \cite{verbin2021ref} and \emph{Blender} dataset from NeRF \cite{mildenhall2021nerf}.}}
\label{tab:ssims}
\end{table*}

\begin{table*}[]
\resizebox{\linewidth}{!}{
\begin{tabular}{l|rrrrrr|rrrrrrrr}
LPIPS $\downarrow$                          & \multicolumn{1}{l}{teapot}   & \multicolumn{1}{l}{toaster}  & \multicolumn{1}{l}{car}      & \multicolumn{1}{l}{ball}     & \multicolumn{1}{l}{coffee}   & \multicolumn{1}{l|}{helmet}  & \multicolumn{1}{l}{chair}    & \multicolumn{1}{l}{lego}     & \multicolumn{1}{l}{materials} & \multicolumn{1}{l}{mic}      & \multicolumn{1}{l}{hotdog}   & \multicolumn{1}{l}{ficus}    & \multicolumn{1}{l}{drums}    & \multicolumn{1}{l}{ship}     \\ \hline
\rowcolor[HTML]{FFFFFF} 
PhySG$^1$                                   & .022                         & .194                         & .091                         & .179                         & .150                         & .089                         & .122                         & .208                         & .182                          & .108                         & .163                         & .144                         & .188                         & .343                         \\
\rowcolor[HTML]{FFFFFF} 
NVDiffRec$^1$                               & .022                         & .180                         & .057                         & .194                         & .097                         & .134                         & \cellcolor[HTML]{FFD9B3}.027 & .037                         & .104                          & .033                         & \cellcolor[HTML]{FFFFB4}.038 & \cellcolor[HTML]{FFD9B3}.030 & .070                         & .208                         \\
\rowcolor[HTML]{FFFFFF} 
NVDiffRecMC$^1$                             & .029                         & .243                         & .086                         & .346                         & .131                         & .215                         & .080                         & .075                         & .096                          & .057                         & .089                         & .076                         & .096                         & .319                         \\
\rowcolor[HTML]{FFB3B3} 
\cellcolor[HTML]{FFFFFF}Ref-NeRF$^2$        & .004                         & .095                         & \cellcolor[HTML]{FFFFFF}.041 & \cellcolor[HTML]{FFFFFF}.059 & \cellcolor[HTML]{FFFFFF}.078 & \cellcolor[HTML]{FFFFB4}.075 & .017                         & \cellcolor[HTML]{FFB3B3}.018 & .022                          & .007                         & .022                         & .019                         & .059                         & \cellcolor[HTML]{FFD9B3}.139 \\
\rowcolor[HTML]{FFFFFF} 
Ours, no integral image                     & .013                         & .285                         & .077                         & .399                         & \cellcolor[HTML]{FFD9B3}.065 & .180                         & .055                         & .031                         & .074                          & .042                         & .051                         & .039                         & .077                         & .180                         \\
\rowcolor[HTML]{FFFFFF} 
Ours, analytical derivative                 & .011                         & .235                         & .053                         & .353                         & .071                         & .118                         & .052                         & .031                         & .048                          & .028                         & .047                         & .043                         & .075                         & .190                         \\
\cellcolor[HTML]{FFFFFF}Ours, single bounce & \cellcolor[HTML]{FFD9B3}.008 & \cellcolor[HTML]{FFFFB4}.114 & \cellcolor[HTML]{FFB3B3}.033 & \cellcolor[HTML]{FFD9B3}.047 & \cellcolor[HTML]{FFB3B3}.063 & \cellcolor[HTML]{FFB3B3}.050 & \cellcolor[HTML]{FFFFB4}.032 & \cellcolor[HTML]{FFB3B3}.018 & \cellcolor[HTML]{FFD9B3}.026  & \cellcolor[HTML]{FFD9B3}.020 & \cellcolor[HTML]{FFD9B3}.034 & \cellcolor[HTML]{FFFFB4}.033 & \cellcolor[HTML]{FFD9B3}.065 & \cellcolor[HTML]{FFB3B3}.135 \\
\rowcolor[HTML]{FFFFFF} 
Ours, no neural                             & \cellcolor[HTML]{FFD9B3}.008 & .115                         & \cellcolor[HTML]{FFFFB4}.039 & \cellcolor[HTML]{FFFFB4}.058 & .071                         & .090                         & .053                         & \cellcolor[HTML]{FFD9B3}.026 & \cellcolor[HTML]{FFFFB4}.036  & .027                         & .045                         & .036                         & .070                         & .161                         \\
\cellcolor[HTML]{FFFFFF}Ours                & \cellcolor[HTML]{FFFFB4}.010 & \cellcolor[HTML]{FFD9B3}.104 & \cellcolor[HTML]{FFD9B3}.034 & \cellcolor[HTML]{FFB3B3}.046 & \cellcolor[HTML]{FFFFB4}.069 & \cellcolor[HTML]{FFD9B3}.055 & \cellcolor[HTML]{FFFFFF}.044 & \cellcolor[HTML]{FFFFB4}.024 & \cellcolor[HTML]{FFD9B3}.026  & \cellcolor[HTML]{FFFFB4}.022 & \cellcolor[HTML]{FFFFFF}.046 & \cellcolor[HTML]{FFFFFF}.044 & \cellcolor[HTML]{FFFFB4}.068 & \cellcolor[HTML]{FFFFB4}.149
\end{tabular}
}
\small{$^1$ requires object masks during training. ~~ $^2$ view synthesis method, not inverse rendering. ~~Red is best, followed by orange, then yellow. }
\caption{\textbf{LPIPS Results on the \emph{Shiny Blender} dataset from Ref-NeRF \cite{verbin2021ref} and \emph{Blender} dataset from NeRF \cite{mildenhall2021nerf}.}}
\label{tab:lpips}
\end{table*}

\begin{table*}[]
\resizebox{\linewidth}{!}{
\begin{tabular}{
>{\columncolor[HTML]{FFFFFF}}l |
>{\columncolor[HTML]{FFFFFF}}r 
>{\columncolor[HTML]{FFFFFF}}r 
>{\columncolor[HTML]{FFFFFF}}r 
>{\columncolor[HTML]{FFFFFF}}r 
>{\columncolor[HTML]{FFFFFF}}r 
>{\columncolor[HTML]{FFFFFF}}r |
>{\columncolor[HTML]{FFFFFF}}r 
>{\columncolor[HTML]{FFFFFF}}r 
>{\columncolor[HTML]{FFFFFF}}r 
>{\columncolor[HTML]{FFFFFF}}r 
>{\columncolor[HTML]{FFFFFF}}r 
>{\columncolor[HTML]{FFFFFF}}r 
>{\columncolor[HTML]{FFFFFF}}r 
>{\columncolor[HTML]{FFFFFF}}r }
$\text{MAE}^\circ \downarrow$ & \multicolumn{1}{l}{\cellcolor[HTML]{FFFFFF}teapot} & \multicolumn{1}{l}{\cellcolor[HTML]{FFFFFF}toaster} & \multicolumn{1}{l}{\cellcolor[HTML]{FFFFFF}car} & \multicolumn{1}{l}{\cellcolor[HTML]{FFFFFF}ball} & \multicolumn{1}{l}{\cellcolor[HTML]{FFFFFF}coffee} & \multicolumn{1}{l|}{\cellcolor[HTML]{FFFFFF}helmet} & \multicolumn{1}{l}{\cellcolor[HTML]{FFFFFF}chair} & \multicolumn{1}{l}{\cellcolor[HTML]{FFFFFF}lego} & \multicolumn{1}{l}{\cellcolor[HTML]{FFFFFF}materials} & \multicolumn{1}{l}{\cellcolor[HTML]{FFFFFF}mic} & \multicolumn{1}{l}{\cellcolor[HTML]{FFFFFF}hotdog} & \multicolumn{1}{l}{\cellcolor[HTML]{FFFFFF}ficus} & \multicolumn{1}{l}{\cellcolor[HTML]{FFFFFF}drums} & \multicolumn{1}{l}{\cellcolor[HTML]{FFFFFF}ship} \\ \hline
PhySG$^1$                     & 6.634                                              & \cellcolor[HTML]{FFFFB4}9.749                       & 8.844                                           & \cellcolor[HTML]{FFB3B3}0.700                    & 22.514                                             & \cellcolor[HTML]{FFB3B3}2.324                       & 18.569                                            & 40.244                                           & 18.986                                                & 26.053                                          & 28.572                                             & 35.974                                            & \cellcolor[HTML]{FFD9B3}21.696                    & 43.265                                           \\
NVDiffRec$^1$                 & \cellcolor[HTML]{FFB3B3}3.874                      & 14.336                                              & 15.286                                          & 5.584                                            & \cellcolor[HTML]{FFB3B3}11.132                     & 20.513                                              & 25.023                                            & 42.978                                           & 26.969                                                & 26.571                                          & 29.115                                             & 38.647                                            & 26.512                                            & 39.262                                           \\
NVDiffRecMC$^1$               & 5.928                                              & 11.905                                              & \cellcolor[HTML]{FFFFB4}8.357                   & 1.313                                            & 18.385                                             & 8.131                                               & 23.469                                            & 42.706                                           & 9.132                                                 & 26.184                                          & 26.470                                             & \cellcolor[HTML]{FFB3B3}34.324                    & 25.219                                            & 41.952                                           \\
Ref-NeRF$^2$                  & 9.234                                              & 42.870                                              & 14.927                                          & 1.548                                            & \cellcolor[HTML]{FFD9B3}12.240                     & 29.484                                              & 19.852                                            & \cellcolor[HTML]{FFB3B3}24.469                   & 9.531                                                 & 24.938                                          & \cellcolor[HTML]{FFFFB4}13.211                     & 41.052                                            & 27.853                                            & 31.707                                           \\
Ours, no integral image       & 10.078                                             & 39.779                                              & 28.744                                          & 45.998                                           & 14.776                                             & 28.550                                              & 20.594                                            & 26.712                                           & 27.462                                                & 29.956                                          & 15.188                                             & 36.543                                            & 32.118                                            & 37.464                                           \\
Ours, analytical derivative   & 6.400                                              & 21.403                                              & 10.685                                          & 21.145                                           & 15.425                                             & 8.800                                               & 17.801                                            & 26.852                                           & \cellcolor[HTML]{FFFFB4}8.960                         & \cellcolor[HTML]{FFB3B3}19.426                  & 14.138                                             & \cellcolor[HTML]{FFFFB4}35.505                    & 27.333                                            & 36.423                                           \\
Ours, single bounce           & 6.343                                              & \cellcolor[HTML]{FFD9B3}7.133                       & \cellcolor[HTML]{FFD9B3}7.746                   & \cellcolor[HTML]{FFFFB4}0.722                    & \cellcolor[HTML]{FFFFB4}12.950                     & \cellcolor[HTML]{FFFFB4}2.401                       & \cellcolor[HTML]{FFB3B3}14.285                    & \cellcolor[HTML]{FFFFB4}26.082                   & \cellcolor[HTML]{FFD9B3}8.315                         & \cellcolor[HTML]{FFD9B3}20.004                  & \cellcolor[HTML]{FFD9B3}10.263                     & 37.498                                            & \cellcolor[HTML]{FFFFB4}22.358                    & \cellcolor[HTML]{FFB3B3}29.771                   \\
Ours, no neural               & \cellcolor[HTML]{FFD9B3}4.508                      & 10.288                                              & 8.388                                           & \cellcolor[HTML]{FFD9B3}0.703                    & 14.745                                             & 5.320                                               & \cellcolor[HTML]{FFFFB4}17.503                    & 28.290                                           & 9.549                                                 & 20.181                                          & 13.356                                             & \cellcolor[HTML]{FFD9B3}35.298                    & 24.651                                            & \cellcolor[HTML]{FFFFB4}30.326                   \\
Ours                          & \cellcolor[HTML]{FFFFB4}5.672                      & \cellcolor[HTML]{FFB3B3}6.660                       & \cellcolor[HTML]{FFB3B3}7.742                   & 0.723                                            & 13.173                                             & \cellcolor[HTML]{FFD9B3}2.395                       & \cellcolor[HTML]{FFD9B3}14.330                    & \cellcolor[HTML]{FFD9B3}25.918                   & \cellcolor[HTML]{FFB3B3}8.101                         & \cellcolor[HTML]{FFFFB4}20.144                  & \cellcolor[HTML]{FFB3B3}10.043                     & 37.405                                            & \cellcolor[HTML]{FFB3B3}21.524                    & \cellcolor[HTML]{FFD9B3}30.152                  
\end{tabular}
}
\small{$^1$ requires object masks during training. ~~ $^2$ view synthesis method, not inverse rendering. ~~Red is best, followed by orange, then yellow. }
\caption{\textbf{MAE Results on the \emph{Shiny Blender} dataset from Ref-NeRF \cite{verbin2021ref} and \emph{Blender} dataset from NeRF \cite{mildenhall2021nerf}.}}
\label{tab:maes}
\end{table*}

\plotscene{ball} 
% See \cref{fig:ball} for the ball scene.

% \plotsceneshort{car} 
% See \cref{fig:car} for the car scene.

% \plotscene{helmet} 
% See \cref{fig:helmet} for the helmet scene.

\plotscene{coffee}

\plotscenetop{teapot} 
%See \cref{fig:teapot} for the teapot scene.

\plotscene{toaster} 
%See \cref{fig:toaster} for the toaster scene.

\plotsceneshort{materials} 
%See \cref{fig:materials} for the materials scene.

\plotscene{drums} 
%See \cref{fig:drums} for the drums scene.

\plotscene{ficus} 
%See \cref{fig:ficus} for the ficus scene.

\plotscenebottom{hotdog} 
%See \cref{fig:hotdog} for the hotdog scene.

\plotscene{mic} 
%See \cref{fig:mic} for the mic scene.

\plotship{ship}
%See \cref{fig:ship} for the ship scene.

\plotscene{lego} 
%See \cref{fig:lego} for the ship scene.

\end{document}
