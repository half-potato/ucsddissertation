% https://tikz.net/zoom/
\newif\ifblackandwhitecycle
\gdef\patternnumber{0}

\pgfkeys{/tikz/.cd,
    zoombox paths/.style={
        draw=orange,
        very thick
    },
    black and white/.is choice,
    black and white/.default=static,
    black and white/static/.style={
        draw=white,
        zoombox paths/.append style={
            draw=white,
            postaction={
                draw=black,
                loosely dashed
            }
        }
    },
    black and white/static/.code={
        \gdef\patternnumber{1}
    },
    black and white/cycle/.code={
        \blackandwhitecycletrue
        \gdef\patternnumber{1}
    },
    black and white pattern/.is choice,
    black and white pattern/0/.style={},
    black and white pattern/1/.style={
            draw=white,
            postaction={
                draw=black,
                dash pattern=on 2pt off 2pt
            }
    },
    black and white pattern/2/.style={
            draw=white,
            postaction={
                draw=black,
                dash pattern=on 4pt off 4pt
            }
    },
    black and white pattern/3/.style={
            draw=white,
            postaction={
                draw=black,
                dash pattern=on 4pt off 4pt on 1pt off 4pt
            }
    },
    black and white pattern/4/.style={
            draw=white,
            postaction={
                draw=black,
                dash pattern=on 4pt off 2pt on 2 pt off 2pt on 2 pt off 2pt
            }
    },
    zoomboxarray inner gap/.initial=5pt,
    zoomboxarray columns/.initial=2,
    zoomboxarray rows/.initial=1,
    zoomboxarray heightmultiplier/.initial=0.5,
    subfigurename/.initial={},
    figurename/.initial={zoombox},
    zoomboxarray/.style={
        execute at begin picture={
            \begin{scope}[
                spy using outlines={%
                    zoombox paths,
                    width=\imagewidth / \pgfkeysvalueof{/tikz/zoomboxarray columns} - (\pgfkeysvalueof{/tikz/zoomboxarray columns} - 1) / \pgfkeysvalueof{/tikz/zoomboxarray columns} * \pgfkeysvalueof{/tikz/zoomboxarray inner gap} -\pgflinewidth,
                    height=\pgfkeysvalueof{/tikz/zoomboxarray heightmultiplier} * (\imageheight / \pgfkeysvalueof{/tikz/zoomboxarray rows} - (\pgfkeysvalueof{/tikz/zoomboxarray rows} - 1) / \pgfkeysvalueof{/tikz/zoomboxarray rows} * \pgfkeysvalueof{/tikz/zoomboxarray inner gap}-\pgflinewidth),
                    magnification=3,
                    every spy on node/.style={
                        zoombox paths
                    },
                    every spy in node/.style={
                        zoombox paths
                    }
                }
            ]
        },
        execute at end picture={
            \end{scope}
     \gdef\patternnumber{0}
        },
        spymargin/.initial=0.5em,
        zoomboxes xshift/.initial=1,
        zoomboxes right/.code=\pgfkeys{/tikz/zoomboxes xshift=1},
        zoomboxes left/.code=\pgfkeys{/tikz/zoomboxes xshift=-1},
        zoomboxes yshift/.initial=0,
        zoomboxes above/.code={
            \pgfkeys{/tikz/zoomboxes yshift=1},
            \pgfkeys{/tikz/zoomboxes xshift=0}
        },
        zoomboxes below/.code={
            \pgfkeys{/tikz/zoomboxes yshift=-1},
            \pgfkeys{/tikz/zoomboxes xshift=0}
        },
        caption margin/.initial=0ex, %
    },
    adjust caption spacing/.code={},
    image container/.style={
        inner sep=0pt,
        at=(image.north),
        anchor=north,
        adjust caption spacing
    },
    zoomboxes container/.style={
        inner sep=0pt,
        at=(image.north),
        anchor=north,
        name=zoomboxes container,
        xshift=\pgfkeysvalueof{/tikz/zoomboxes xshift}*(\imagewidth+\pgfkeysvalueof{/tikz/spymargin}),
        yshift=\pgfkeysvalueof{/tikz/zoomboxes yshift}*(\imageheight+\pgfkeysvalueof{/tikz/spymargin}+\pgfkeysvalueof{/tikz/caption margin}),
        adjust caption spacing
    },
    calculate dimensions/.code={
        \pgfpointdiff{\pgfpointanchor{image}{south west} }{\pgfpointanchor{image}{north east} }
        \pgfgetlastxy{\imagewidth}{\imageheight}
        \global\let\imagewidth=\imagewidth
        \global\let\imageheight=\imageheight
        \gdef\columncount{1}
        \gdef\rowcount{1}
        \gdef\zoomboxcount{1}
    },
    image node/.style={
        inner sep=0pt,
        name=image,
        anchor=south west,
        append after command={
            [calculate dimensions]
            node [image container,subfigurename=\pgfkeysvalueof{/tikz/figurename}-image] {\phantomimage}
            node [zoomboxes container,subfigurename=\pgfkeysvalueof{/tikz/figurename}-zoom] {\phantomimage}
        }
    },
    color code/.style={
        zoombox paths/.append style={draw=#1}
    },
    connect zoomboxes/.style={
    spy connection path={\draw[draw=none,zoombox paths] (tikzspyonnode) -- (tikzspyinnode);}
    },
    help grid code/.code={
        \begin{scope}[
                x={(image.south east)},
                y={(image.north west)},
                font=\footnotesize,
                help lines,
                overlay
            ]
            \foreach \x in {0,1,...,9} {
                \draw(\x/10,0) -- (\x/10,1);
                \node [anchor=north] at (\x/10,0) {0.\x};
            }
            \foreach \y in {0,1,...,9} {
                \draw(0,\y/10) -- (1,\y/10);                        \node [anchor=east] at (0,\y/10) {0.\y};
            }
        \end{scope}
    },
    help grid/.style={
        append after command={
            [help grid code]
        }
    },
}

\newcommand\phantomimage{%
    \phantom{%
        \rule{\imagewidth}{\imageheight}%
    }%
}
\newcommand\zoombox[2][]{
    \begin{scope}[zoombox paths]
        \pgfmathsetmacro\xpos{
            (\columncount-1)*(\imagewidth / \pgfkeysvalueof{/tikz/zoomboxarray columns} + \pgfkeysvalueof{/tikz/zoomboxarray inner gap} / \pgfkeysvalueof{/tikz/zoomboxarray columns} ) + \pgflinewidth
        }
        \pgfmathsetmacro\ypos{
            (\rowcount-1) * (\imageheight / \pgfkeysvalueof{/tikz/zoomboxarray rows} + \pgfkeysvalueof{/tikz/zoomboxarray inner gap} / \pgfkeysvalueof{/tikz/zoomboxarray rows} ) + 0.5*\pgflinewidth
        }
        \edef\dospy{\noexpand\spy [
            #1,
            zoombox paths/.append style={
                black and white pattern=\patternnumber
            },
            every spy on node/.append style={#1},
            x=\imagewidth,
            y=\imageheight
        ] on (#2) in node [anchor=north west] at ($(zoomboxes container.north west)+(\xpos pt,-\ypos pt)$);}
        \dospy
        \pgfmathtruncatemacro\pgfmathresult{ifthenelse(\columncount==\pgfkeysvalueof{/tikz/zoomboxarray columns},\rowcount+1,\rowcount)}
        \global\let\rowcount=\pgfmathresult
        \pgfmathtruncatemacro\pgfmathresult{ifthenelse(\columncount==\pgfkeysvalueof{/tikz/zoomboxarray columns},1,\columncount+1)}
        \global\let\columncount=\pgfmathresult
        \ifblackandwhitecycle
            \pgfmathtruncatemacro{\newpatternnumber}{\patternnumber+1}
            \global\edef\patternnumber{\newpatternnumber}
        \fi
    \end{scope}
}





\newcommand{\plottrim}[1]{%
  \adjincludegraphics[trim={{0\width} {0.2\height} {0\width} {0.2\height}}, clip, width=0.24\linewidth]{#1}%
}

\definecolor{col1}{HTML}{e0d291}
\definecolor{col2}{HTML}{d66079}
\newcommand\plotzoomed[1]{
    \raisebox{-0.22\height}{
    \begin{tikzpicture}[
    zoomboxarray,
    zoomboxes below,
    connect zoomboxes,
    zoombox paths/.append style={thick}]
        \node[image node]{\plottrim{#1}};
        \zoombox[magnification=4,color code=col1]{0.28,0.670}
        \zoombox[magnification=4,color code=col2]{0.85,0.35}
        %
    \end{tikzpicture}
    }
}


\newcommand{\plotteaser}[1]{
\twocolumn[{
\renewcommand\twocolumn[1][]{#1}
\maketitle
\begin{center}
%\includegraphics[width=\linewidth,height=190px]{example-image-a}
\begin{tabular}{l@{~}c@{}c@{}c@{}} \vspace{-1.3cm}
\rotatebox{90}{~~~~~~~~~~~~~~~~Ours}
& \plotzoomed{images/#1/final.png}
& \plotzoomed{images/#1/normals.png}
& \plotall{images/#1/mapped_pano.png} \\ \vspace{-1cm}
\rotatebox{90}{~~~~~~~~~Ground Truth}
& \plotzoomed{images/#1/gt_final.png}
& \plotzoomed{images/#1/gt_normals.png}
& \plotall{images/#1/gt_pano.png} \\
& Novel View & Surface Normals & Environment Map
\end{tabular}
%\captionof{figure}{\modelname{} (top) recovers materials (\bsdf{}), geometry (density), and illumination (environment) from calibrated images of a scene, closely matching the ground truth (bottom). Here we show results on the \emph{#1} scene: a rendered novel view (left), density-based surface normals (middle), and environment map (right).}
\captionof{figure}{Our method (top) recovers materials, geometry, and illumination that closely resemble the ground truth (bottom), optimizing directly from calibrated images of a scene. Here we show results on the \emph{#1} scene from NeRF~\cite{mildenhall2021nerf}: a rendered novel view (left), surface normals (middle), and environment map illumination (right). Insets show high-fidelity reflections, including interreflections, as well as accurate geometry even in concave regions. \todo{bottom of our surface normals are getting cut off}}
\label{fig:#1_teaser}
\end{center}
}]
}


% version without zoom-in insets
% \newcommand{\plotteaser}[1]{
% \twocolumn[{
% \renewcommand\twocolumn[1][]{#1}
% \maketitle
% \begin{center}
% %\includegraphics[width=\linewidth,height=190px]{example-image-a}
% \begin{tabular}{cccc}
% \rotatebox{90}{\modelname{}}
% & \plotall{images/#1/final.png}
% & \plotall{images/#1/normals.png}
% & \plotall{images/#1/mapped_pano.png} \\
% \rotatebox{90}{~~~~~~~~~Ground Truth}
% & \plotall{images/#1/gt_final.png}
% & \plotall{images/#1/gt_normals.png}
% & \plotall{images/#1/gt_pano.png} \\
% & Novel View & Density-Based Surface Normals & Environment Map
% \end{tabular}
% %\captionof{figure}{\modelname{} (top) recovers materials (\bsdf{}), geometry (density), and illumination (environment) from calibrated images of a scene, closely matching the ground truth (bottom). Here we show results on the \emph{#1} scene: a rendered novel view (left), density-based surface normals (middle), and environment map (right).}
% \captionof{figure}{Our method (top) recovers materials, geometry, and illumination from calibrated images of a scene, that closely resemble the ground truth (bottom). Here we show results on the \emph{#1} scene from~\cite{mildenhall2021nerf}: a rendered novel view (left), surface normals (middle), and environment map illumination (right).}
% \label{fig:#1_teaser}
% \end{center}
% }]
% }